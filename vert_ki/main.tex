\documentclass[ngerman]{scrartcl}
\usepackage{amsmath,amsthm,amssymb}
\usepackage[T1]{fontenc}
\usepackage[utf8]{inputenc}
\usepackage{lmodern}
\usepackage{graphicx}

\usepackage{hyperref}

\title{Distributed Artificial Intelligence \\ WiSe 19/20}
\author{Benedikt Lüken-Winkels}
\begin{document}

\maketitle
\tableofcontents
\newpage


\section{Motivation}
Where can DAI be applied?
\begin{itemize}
    \item Smart Factory: How can processes be designed to coordinate and communicate with each other (in Webservices, IoT)?
    \item Autonomous Control: Switch from hierarchical, centralized structure to distributed, decentralized control of processes (Package delivery).
\end{itemize}

\section{Agents and agent interaction}
An agent is the basic building block fo an AI system.
\subsection*{Properties of an intelligent agent}
A weak notion of Agency
\begin{itemize}
    \item Autonomy: makes own decisions, env influences the decisions
    \item Reactivity: perceives env and adapts behaviour
    \item Pro-activity: has goals, takes initiative to achieve them
    \item Social ability: interaction (eg exchange info, ask for help,...)
\end{itemize}

\subsection*{Agent interactions with the environment}
Perception (Sensors) $ \rightarrow $ Decision $ \rightarrow $ Action (Effectors). Reactive Agents (Decide according to inputs from env and try to match conditions). Environment $ E = {e_0, e_1, ...} $ is a set of env states. Agent actions $ Ac = {a, a', ...} $ is a set of all actions that changes the Environment state . $ Ac_i $ is the set of actions the Agent $ i $ can perform. Agent Function Ag $ Ag: E \rightarrow Ac $. 
\begin{itemize}
    \item $ R $
    \item $ R^{Ac} $
    \item $ R^E $
\end{itemize} 
\paragraph{Refined Ag = <see, action, next>} Perception, Internal state transition, Action
\begin{itemize}
    \item see: filter env to perception
    \item next: current state and perception to new state
    \item action: map internal state to actions
\end{itemize}

\subsection*{Communication between Agents}
When agents have common goals or intentions they should interact, eg transport service.
\subsubsection*{Message Passing} Speech-Act Theory (augment communication with info about how to interpret messages):
\begin{itemize}
    \item Speaker: Wants hearer to believe in info. Chooses language. Forms message.
    \item Hearer: Perceives message (w/ disturbance). Analyses possible meanings. Interprets and chooses a meaning. Decides to belive the info or rejects
\end{itemize}
Three parts of a Speech-Act:
\begin{itemize}
    \item Locutionary act: meaningful words
    \item Illocutionary act: intention
    \item Perlocutionary act: effect of the intention 
\end{itemize}
Classification:
\begin{itemize}
    \item Representatives: notification
    \item Directives: request, order
    \item Commissives: promise
    \item Expressives:
    \item Declarations: directly changes the env
\end{itemize}




\section{Coordination mechanisms and mechanism design}




\section{Deliberative agents and knowledge representaion}




\section{Norms, trust and reputation into control intelligent distributed systems}




\section{Distributed reinforcement learning}



\end{document}
