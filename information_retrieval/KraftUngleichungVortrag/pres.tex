%\documentclass[aspectratio=169]{beamer}
\documentclass{beamer}
%\geometry{paperwidth=140mm,paperheight=105mm}
\usetheme{metropolis}           % Use metropolis theme
%\usetheme{CambridgeUS}
\usepackage{ngerman}
\usepackage[utf8]{inputenc}
\title{Kraft-McMillan-Ungleichung}
\subtitle{Verlustfreie Kompression}
\date{12. Juni 2018}
\author{Benedikt L{\"u}ken-Winkels}
\institute{Universit{\"a}t Trier}

%Matheumgebung
\usepackage{amsthm}

%Tabelle neben Inhalt
\usepackage{caption,stackengine}

%Pause ausgrauen
\setbeamercovered{transparent}

\usepackage{graphicx}
%Footer
\setbeamertemplate{footline}
{
  \leavevmode%
  \hbox{%
  \begin{beamercolorbox}[wd=.4\paperwidth,ht=2.25ex,dp=1ex,center]{author in head/foot}%
    \usebeamerfont{author in head/foot}Benedikt L{\"u}ken-Winkels
  \end{beamercolorbox}%
  \begin{beamercolorbox}[wd=.6\paperwidth,ht=2.25ex,dp=1ex,center]{title in head/foot}%
    \usebeamerfont{title in head/foot}\insertshorttitle\hspace*{3em}
    \insertframenumber{} / \inserttotalframenumber\hspace*{1ex}
  \end{beamercolorbox}}%
  \vskip0pt%
}
%Farbe für Block
\definecolor{amber}{rgb}{1.0, 0.75, 0.0}
\setbeamercolor{block title}{use=text,
    fg=amber,
    bg=gray}
\setbeamercolor{block body}{use={block title , text},
    fg=text.fg,
    bg=lightgray}
     
    
    
\makeatletter    
    
\mode<presentation>{}
\begin{document}
\nocite{*}
\maketitle
\section{Bedeutung}
\begin{frame}{Bedeutung}
Bedingung für die Dekodierbarkeit eines Codes\pause
\begin{itemize}
\item[ $\rightarrow$] Code nur eindeutig dekodierbar, wenn die Schlüssellängen die Kraft-McMillan-Ungleichung erfüllen.\pause
\item[$\rightarrow$] Erfüllen die Schlüssellängen die Kraft-McMillan-Ungleichung, existiert eine eindeutige Codierung.\pause
\item[$\Rightarrow$] Für jede Schlüssellänge gibt es einen eindeutig dekodierbaren Präfix-Code.
\end{itemize}

\end{frame}

\section{Definition}
\begin{frame}{Definition}

\begin{center}
\begin{minipage}{.65\textwidth}
\begin{block}{Erinnerung: Code}
Abbildung von Worten aus $X$ auf \textit{Codeworte} aus $\Sigma^{*}$ \pause
\begin{align*}
\text{Code } C:X \rightarrow \Sigma^{*}
\end{align*}
\end{block}
\end{minipage}
\end{center}

\end{frame}

\begin{frame}{Definition}

\begin{center}
\begin{minipage}{.9\textwidth}
\begin{block}{Erinnerung: Präfixcode}
Eindeutig dekodierbarer Code bei dem kein Codewort ein Präfix eines anderen Codewortes ist.\pause
\begin{align*}
Beispiel\ 1&: \{0, 01, 11\} \pause \textbf{ Kein Präfixcode}\\ \pause
Beispiel\ 2&: \{10, 01, 11\} \pause \textbf{ Präfixcode}
\end{align*}
\end{block}
\end{minipage}
\end{center}

\end{frame}

\begin{frame}{Definition}

\begin{center}
\begin{minipage}{.9\textwidth}
\begin{block}{Kraft-McMillan-Ungleichung}
Sei Code $C:X \rightarrow \Sigma^{*}$ über einem Alphabet $\Sigma$ mit $|\Sigma| = q$, Länge der Codewörter $ l_{1},...,l_{n}$ und $n$ Wörtern ein eindeutig dekodierbarer Präfix(freier)-Code, dann gilt: \pause
\begin{align*}
KM(C)&= \sum_{i=1}^{n} \frac{1}{q^{l_{i}}} \leq 1
\end{align*}
\end{block}
\end{minipage}
\end{center}

\end{frame}

\section{Anwendung}

\begin{frame}{Anwendung}
\begin{center}
\begin{minipage}{.8\textwidth}
\begin{block}{Beispiel}
Alphabet $\Sigma$, Größe des Alphabets der Codierung $q$, abzubildende Wörter $X$\pause
\begin{align*}
\Sigma & = \{0,1\} \\
& \Rightarrow q = 2\\
X & = \{a, b, c, d\}
\end{align*}
\end{block}
\end{minipage}
\end{center}

\end{frame}

\begin{frame}{Anwendung}
\begin{minipage}{0.49\textwidth}
\begin{table}[htb]
\caption{Beispiel für eindeutige Codierung\label{tab:example1}}
\vspace*{1em}
\centering

\bgroup
\def\arraystretch{1.3}%  1 is the default, change whatever you need

\begin{tabular}[c]{l|l|l}
	
	\multicolumn{1}{c|}{\textbf{$X$}} & 
	\multicolumn{1}{c|}{\textbf{$C(x)$}} & 
	\multicolumn{1}{c}{\textbf{$l_{C(x)}$}} \\ 
	
	\hline

	a & 110 & 3\\
	b & 111 & 3\\
	c & 0 & 1\\
	d & 10 & 2\\
	
\end{tabular}
\egroup

\end{table}

\end{minipage}\pause
\hfill
\begin{minipage}{0.49\textwidth}
\begin{align*}
KM(C) &= \sum_{i=1}^{n} \frac{1}{2^{l_{i}}} \\\pause
& = \frac{1}{2^{3}} + \frac{1}{2^{3}} + \frac{1}{2^{1}} + \frac{1}{2^{2}}\\\pause
& = 1
\end{align*}

\end{minipage}



\end{frame}

\begin{frame}{Anwendung}

\begin{minipage}{0.49\textwidth}

\begin{table}[htb]
\caption{Beispiel für nicht-eindeutige Codierung\label{tab:example2}}
\vspace*{1em}
\centering

\bgroup
\def\arraystretch{1.3}%  1 is the default, change whatever you need

\begin{tabular}[c]{l|l|l}
	
	\multicolumn{1}{c|}{\textbf{$X$}} & 
	\multicolumn{1}{c|}{\textbf{$C(x)$}} & 
	\multicolumn{1}{c}{\textbf{$l_{C(x)}$}} \\ 
	
	\hline

	a & 11 & 2\\
	b & 111 & 3\\
	c & 0 & 1\\
	d & 10 & 2\\
	
\end{tabular}
\egroup

\end{table}

\end{minipage}\pause
\hfill
\begin{minipage}{0.49\textwidth}
\begin{align*}
KM(C) &= \sum_{i=1}^{n} \frac{1}{2^{l_{i}}} \\\pause
& = \frac{1}{2^{2}} + \frac{1}{2^{3}} + \frac{1}{2^{1}} + \frac{1}{2^{2}}\\\pause
& > 1
\end{align*}
\end{minipage}

\end{frame}

\section{Zusammenfassung}

\begin{frame}{Zusammenfassung}
\begin{minipage}{0.39\textwidth}
\begin{align*}
& \sum_{i=1}^{n} \frac{1}{p^{l_{i}}} \leq 1\\
\end{align*}
\end{minipage}\pause
\hfill
\begin{minipage}{0.6\textwidth}
\begin{itemize}
\item Obere Schranke der Summe als 'Budget'\pause
\begin{itemize}
\item[$\rightarrow$] je länger die Codeworte, desto kleiner die Summe\pause
\item[$\rightarrow$] nicht viele kurze Codeworte verfügbar.\pause
\end{itemize}
\item Beispiel: Morsecode
\end{itemize}

\end{minipage}

\end{frame}
\begin{frame}[allowframebreaks]
        \frametitle{Quellen}
         % nur angeben, wenn auch nicht im Text zitierte Quellen 
           % erscheinen sollen
\bibliographystyle{ieeetran}
        \bibliography{sources}
\end{frame}

\end{document}


%========================
%
% END
%
%========================
