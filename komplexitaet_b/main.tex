\documentclass[ngerman]{scrartcl}
\usepackage{amsmath,amsthm,amssymb}
\usepackage[T1]{fontenc}
\usepackage[utf8]{inputenc}
\usepackage{lmodern}
\usepackage{graphicx}

\usepackage{hyperref}

\title{Komplexitätstheorie B \\ WiSe 19/20}
\author{Benedikt Lüken-Winkels}
\begin{document}

\maketitle
\tableofcontents
\newpage

\section{Motivation}
Ziel ist die eindeutige Zuweisung einer eines Problems zu einer Komplexitätsklasse anhand von geeigneten Maschinenmodellen. \textbf{Probleme} sind Entscheidungsprobleme bzw. Mengen von Ja-Instanzen. 

\subsection{Probleme aus Formalen Sprachen (Wortproblem)}
\paragraph{$ P_1 $} eine Phrasenstrukturgrammatik G. Gilt $ L(G) \neq \emptyset $ ? 


\paragraph{$ P_2 $} eine monotone Gammatik G. Gilt $ L(G) \neq \emptyset $ ? 


\paragraph{$ P_3 $} eine kontextfreie Grammatik G. Gilt $ L(G) \neq \emptyset $ ? \\
Ist entscheidbar, weil

\paragraph{$ P_4 $} eine zwei kontextfreie Grammatiken $ G_1, G_2 $. Gilt $ L(G_1) \cap L(G_2) \neq \emptyset $ ? 


\paragraph{$ P_5 $} eine zwei rechtslineare Grammatiken $ G_1, G_2 $. Gilt $ L(G_1) \cap L(G_2) \neq \emptyset $ ?  




\section{Allgemeines}
\subsection{Spezielles Halteproblem $ K $}
\begin{itemize}
    \item Eingabe: Kodierung einer TM
    \item Ja-Instanzen: Menge aller Kodierungen von TMs, die angewendet auf sicht selbst halten.
\end{itemize}

\subsection{Allgemeines Halteproblem $ H $}
\begin{itemize}
    \item Einabe: Paar (Kodierung einer TM M die eine Grammatik realisiert, Wort)
    \item Ja-Instanzen: Das Wort gehört zu M
\end{itemize}

\subsection{Grammatiken}
\begin{itemize}
    \item Phrasenstrukturgrammatik: Typ-0 Grammatik
    \item Monoton (kontextsensitiv): Typ-1 Grammatik
    \begin{itemize}
        \item Rechte Seite darf beim Ersetzen nicht kürzer, als die linke Seite sein.
    \end{itemize}
    \item Kontextfrei: Typ-2 Grammatik
    \begin{itemize}
        \item Jede Regel hat genau ein NT-Symbol auf der linken Seite
        \item Jede Regel hat eine nicht-leere Folge von NT- oder T-Symbolen auf der rechten Seite
    \end{itemize}
    \item Rechtslinear: Äquivalent zu den Typ-3 Grammatiken 
    \begin{itemize}
        \item Höchstens ein NT-Symbol auf der rechten Seite
        \item Das NT-Symbol darf nur am Ende der rechten Seite stehen
    \end{itemize}
    \item Regulär: Typ-3 Grammatik
    \begin{itemize}
        \item Rechte Seite nur das leere Wort, ein oder mehrere Terminalsymbole oder ein oder mehrere T-Symbole gefolgt von einem einzigen NT sein.
    \end{itemize}
\end{itemize}



\end{document}
