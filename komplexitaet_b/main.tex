\documentclass[ngerman]{scrartcl}
\usepackage{amsmath,amsthm,amssymb}
\usepackage[T1]{fontenc}
\usepackage[utf8]{inputenc}
\usepackage{lmodern}
\usepackage{graphicx}

\usepackage{hyperref}

\title{Komplexitätstheorie B \\ WiSe 19/20}
\author{Benedikt Lüken-Winkels}
\begin{document}

\maketitle
\tableofcontents
\newpage

\section{Wiederholung Berechenbarkeit und Komplexität}
Ziel ist die eindeutige Zuweisung einer eines Problems zu einer Komplexitätsklasse anhand von geeigneten Maschinenmodellen. \textbf{Probleme} sind Entscheidungsprobleme bzw. Mengen von Ja-Instanzen. 

\subsection{Probleme aus Formalen Sprachen (Nicht-leerheits-Problem)}
\paragraph{$ P_1 $} eine Phrasenstrukturgrammatik G. Gilt $ L(G) \neq \emptyset $ ? 


\paragraph{$ P_2 $} eine monotone Gammatik G. Gilt $ L(G) \neq \emptyset $ ? 


\paragraph{$ P_3 $} eine kontextfreie Grammatik G. Gilt $ L(G) \neq \emptyset $ ? 
Ist entscheidbar, mit dem CYK-Algorithmus.

\subsection{Reduktion}


\subsection{Kodierung}
Abhängig von der Kodierung ändert sich die Problemgröße. Zulässige Kodierungen müssen berechenbar sein: Man mann einer Kodierung einer TM zB kein Halteproblembit einfügen, das bezeichnet, ob es eine Ja-Instanz ist.
\paragraph{Knapsack} 
\begin{itemize}
    \item In: $ a_1,..,a_n; p_1,...,p_n; a,c \in \mathbb{N} $
    \item $ \exists I \subseteq \{ 1,...,n \} $ mit Summe von $ p_i \geq p$ und Summe von $ a_i \leq a$
\end{itemize}

\subsection{Turingmaschinen}
\begin{itemize}
    \item Konfiguration: Randbegrenzer, beliebige Zeichen aus Alphabet, Zustandsübergänge
    \item Anfangskonfiguration: $ \$ s_0 w\ square $ ist keine Konfiguration mit $ w = \lambda $ ($ \lambda $ wird als blank definiert)
    \item Konfigurationsübergang: (beachte nach links nicht erweiterbar) Tupel aus 
    \item Akzeptierte Sprache: Menge aller Eingabe Wörter für die wir beim Startsymbol mit beliebig vielen Schritten in einen Endzustand kommen.
\end{itemize}

\paragraph{Deterministische TM} Für jedes Tupel aus Zuststand und Alphabet gibt es höchstens ein Tripel aus Zustandübergängen

\paragraph{Mehrband TM} Simultanes Arbeiten auf mehreren Bändern: Arbeitsband ist getrennt vom Eingabeband. Simulation von Mehrband TMs auf Einband TM durch \textbf{Spuren} und Aufblähung des Alphabets. Aufwand für eine k-Bandmaschine: 2k 

\subsection{NP-Schwere}
\paragraph{Generische Ausgangsproblem für NP-Schwere} ist das Short NTM Acceptance. 
\paragraph{SAT} Formel in Aussagelogik. Gibt es eine Belegung der Variablen, sodass die Formel erfüllt ist? Messung der Laufzeit eines det Algorithmus zB $ 2^n $


\section{Parametrisierte Probleme}
Messen der Komplextität eines Problems nicht nur in Bitlänge, sondern Ressourcenbedarf. Parametrisiertes Problem $ (P, \kappa) $ mit $ \kappa $ als Parametrisierungsfunktion (Anzahl der Klauseln bei SAT).  Für $ (P, \kappa) \in FTP$ gibt es einen det Algorithmus (DTM) mit Zeit $ O(f(\kappa(x))|x|^c) $.
\paragraph{Parametrisierte Reduktionen}  Der Parameter $ \kappa_1(x) $ für $ P_2 $ soll durch eine Abbildung g gedeckelt werden sollen.

\subsection{Typische FTP-Probleme}
\paragraph{MaxSAT} Gegeben eine Formel in KNF und eine nat Zahl k. Finde eine Belegung, sodass $ \geq k $ Klauseln erfüllt werden. Ist NP-schwer wegen Reduktion auf KNF-SAT mit k = Anzahl der Klauseln. Belegt man alle Klauseln mit 0 oder 1 ist jeweils die Hälte aller Klauseln erfüllt. Das bedeutet, dass bei k < m/2 automatisch eine Ja-Instanz vorliegt. 

\paragraph{Vertex Cover}

\section{Die Klasse W[1]}
Short NTM Acceptance lässt sich nicht mit DTM in Polynomzeit lösen. Beim Versuch der Parametrisierung

\subsection{Probleme aus W[1]}
\paragraph{Short NTM Word Acceptance} Akzeptiert die TM M das Wort w in k Schritten
\paragraph{Independent Set} In einem Graphen suche eine Menge von k Knoten, die keine gemeinsame Kante haben.  




%===============================
%
%
%
%
%===============================
\section{Allgemeines}
\subsection{Simulation von NTM auf DTM}
Grad des Nichtdeterminismus hängt von der Eingabe der NTM ab.


\subsection{Satz von Cook}

\subsection{Spezielles Halteproblem $ K $}
\begin{itemize}
    \item Eingabe: Kodierung einer TM
    \item Ja-Instanzen: Menge aller Kodierungen von TMs, die angewendet auf sicht selbst halten.
\end{itemize}

\subsection{Allgemeines Halteproblem $ H $}
\begin{itemize}
    \item Einabe: Paar (Kodierung einer TM M die eine Grammatik realisiert, Wort)
    \item Ja-Instanzen: Das Wort gehört zu M
\end{itemize}

\subsection{Grammatiken}
\begin{itemize}
    \item Phrasenstrukturgrammatik: Typ-0 Grammatik
    \item Kontextsensitiv: Typ-1 Grammatik
    \item Monoton
    \begin{itemize}
        \item Rechte Seite darf beim Ersetzen nicht kürzer, als die linke Seite sein.
    \end{itemize}
    \item Kontextfrei: Typ-2 Grammatik
    \begin{itemize}
        \item Jede Regel hat genau ein NT-Symbol auf der linken Seite
        \item Jede Regel hat eine nicht-leere Folge von NT- oder T-Symbolen auf der rechten Seite
    \end{itemize}
    \item Rechtslinear: Äquivalent zu den Typ-3 Grammatiken 
    \begin{itemize}
        \item Höchstens ein NT-Symbol auf der rechten Seite
        \item Das NT-Symbol darf nur am Ende der rechten Seite stehen
    \end{itemize}
    \item Regulär: Typ-3 Grammatik
    \begin{itemize}
        \item Rechte Seite nur das leere Wort, ein oder mehrere Terminalsymbole oder ein oder mehrere T-Symbole gefolgt von einem einzigen NT sein.
    \end{itemize}
\end{itemize}

\subsection{Reduktion}
\paragraph{Karp-Reduktion}

\paragraph{Many-One-Reduktion}

\subsection{Übersetzung von det TM auf nicht-det TM}

\subsection{Dynamisches Programmieren}
Verwenden vorheriger Ergebnisse während der Laufzeit ($ \Rightarrow $ Rekursive Algorithmen)




%===============================
%
%
%
%
%===============================
\section{Aufgaben}

\paragraph{Aufgabe 1: Sind diese Sprachen entscheidbar?} 

\subparagraph{$ P_4 $} eine zwei kontextfreie Grammatiken $ G_1, G_2 $. Gilt $ L(G_1) \cap L(G_2) \neq \emptyset $ ? 
Simulation von Berechnungen von TMs

\subparagraph{$ P_5 $} eine zwei rechtslineare Grammatiken $ G_1, G_2 $. Gilt $ L(G_1) \cap L(G_2) \neq \emptyset $ ?  
\begin{itemize}
    \item Reguläre Sprachen sind Schnittabgeschlossen. 
    \item Es gibt einen Algorithmus der aus dem Schnitt von 2 det TMs eine det TM liefert. $ \Rightarrow $ entscheidbar.
\end{itemize}

\paragraph{Aufgabe 2: Ist das Wortproblem (Folie 9) aufzählbar?} $ M \in K \Leftrightarrow f(M) = (G_K, c(M)) \in W $. $G_K$ ist die Grammatik, die Kodierungen von TMs aus K beschreibt. Für $G \in W$ gilt, $(G, w) mit\ w \in L(G)$. Das heißt, eine TM M ist eine Ja-Instanz für K, sobald $L(G_K)$ eine Kodierung ($c(M)$) von M erzeugen kann.
\paragraph{Aufgabe 3: Wieso Polynomzeit für Unary Knapsack?} Summe der Eingabe kodiert in Bits ergibt 
\begin{itemize}
    \item Verwende Datenstruktur Tabelle (Profit P/ Kosten A) und fülle sie mit 1 oder 0, wenn der Profit $ p_i $ mit Kosten $ a_j $ erreichbar ist 
\end{itemize}
\paragraph{Aufgabe 4: Formalisierung der Kopfbewegungen einer TM (Folie 13)}
Vorsicht bei Linksbewegung, wegen $ \$ $
\paragraph{Aufgabe 5: Formalisierung einer DTM und Mehrband TM}


\paragraph{Aufgabe 6: Ist die Zahlenkodierung/Einband-/Mehrband-TM bei Short NTM Acceptance wesentlich?}
Wenn die Schrittzahlbegrenzung k in unär kodiert ist. Besonderheit bei Mehrband: 

\paragraph{Aufgabe 7: Wie schnell lässt sich SAT lösen?}
Bei KNF-SAT kann die Klauselanzahl eine Messgröße sein. Lässt sich das auf SAT übertragen?

\paragraph{Aufgabe 8: Lösen von VC mit Short NTM Acceptance (zwei many-one Polynomzeitreduktionen)} Erstellen einer NTM und dann Reduktion mit Short NTM Acceptance (Rechenteppich) 
\begin{itemize}
    \item 1. Reduktion:
    \item 2. Reduktion:
\end{itemize}

\paragraph{Aufgabe 9: Zeigen, dass VC $\leq$ Short NTM Acceptance gilt. und umgekehrt}


\paragraph{Aufgabe 10: Zeigen, dass $ \MinSAT \leq_{FTP} VC $ (Folie 25)}
\paragraph{Aufgabe 11}
\paragraph{Aufgabe 12}
\paragraph{Aufgabe 13}
\paragraph{Aufgabe 14}
\paragraph{Aufgabe 15}








\end{document}
