\documentclass[aspectratio=169]{beamer}
\usetheme{metropolis}           % Use metropolis theme
%\usetheme{CambridgeUS}
\title{Simulation von Turing Maschinen}
\subtitle{Unm{\"o}glichkeit einer universellen Turing Maschine mit nur einem Zustand}
\date{12. Juni 2018}
\author{Benedikt L{\"u}ken-Winkels}
\institute{Universit{\"a}t Trier}


%Deutsche Symbole möglich
\usepackage{ngerman}
\usepackage[utf8]{inputenc}

%Matheumgebung
\usepackage{amsthm}

\usepackage{graphicx}

%Pause ausgrauen
\setbeamercovered{transparent}

%Footer
\setbeamertemplate{footline}
{
  \leavevmode%
  \hbox{%
  \begin{beamercolorbox}[wd=.4\paperwidth,ht=2.25ex,dp=1ex,center]{author in head/foot}%
    \usebeamerfont{author in head/foot}Benedikt L{\"u}ken-Winkels
  \end{beamercolorbox}%
  \begin{beamercolorbox}[wd=.6\paperwidth,ht=2.25ex,dp=1ex,center]{title in head/foot}%
    \usebeamerfont{title in head/foot}\insertshorttitle\hspace*{3em}
    \insertframenumber{} / \inserttotalframenumber\hspace*{1ex}
  \end{beamercolorbox}}%
  \vskip0pt%
}
%Farbe für Block
\definecolor{amber}{rgb}{1.0, 0.75, 0.0}
\setbeamercolor{block title}{use=text,
    fg=amber,
    bg=gray}
\setbeamercolor{block body}{use={block title , text},
    fg=text.fg,
    bg=lightgray}
    
\makeatletter    
    
\mode<presentation>{}
\begin{document}
\maketitle

\nocite{*}

\begin{frame}{Begriffserkl{\"a}rung}
\begin{block}{Beschreibungsnummer} \pause
\begin{itemize}
\item auch Gödelnummer 
\item Kodierung einer Turing Maschine auf einer universellen Turing Maschine (UTM)
\end{itemize}
\end{block}
\end{frame}


\begin{frame}
\begin{block}{Voraussetzungen}
TM $M = ( Q, \Sigma, \Gamma, \delta, q, \square, F)$ mit \pause
\begin{itemize}
\item $Q = \{q\}$,
\item einem Band, \pause
\item endl. Beschreibungsnummer einer TM, die eine bel. berechenbare (irrationale) Zahl berechnen kann auf dem Band.
\end{itemize}
\end{block}
\end{frame}

\begin{frame}{Beweis}
\begin{block}{Idee}
Berechnung der Ziffern von $\sqrt{2}$ durch $M$ ist nicht möglich, wenn entweder \pause
\begin{itemize}
\item[*$_{1}$] eine endliche Anzahl von Zellen auf dem Band verarbeitet wurden und auf dem Rest des Bandes das gleiche Symbol steht , oder \pause
\item[*$_{2}$] außer einer endlichen Anzahl von Zellen wird jede Zelle unendlich oft geändert.
\end{itemize}
\end{block}
\end{frame}

\begin{frame}{Beweis (1)}
\begin{align*}
& B:= \text{ Zellen des Bandes mit der Beschreibungsnummer der TM, die } \sqrt{2} \text{ berechnet} \\
& A:= \text{ Unendliche Folge von Blanksymbolen, links von } B \\
& C:= \text{ Unendliche Folge von Blanksymbolen, rechts von } B \\
\end{align*}
\end{frame}

\begin{frame}{Beweis (2)}
$\text{Bei } \delta(q, \square) \text{ muss } M \overbrace{\text{stehen bleiben}}^{\text{Fall 1}} \text{ oder } \overbrace{\text{nach links/rechts gehen.}}^{\text{Fall 2}}$ \pause
\begin{align*}
\text{Fall 1:} &\ \text{alles außer einem endl. Teil des Bandes hat das Blanksymbol}\Rightarrow *_{1} \\ \pause
\text{Fall 2.1:} &\ (a)\ A \text{ wird nie erreicht oder }\pause (b) \text{ alle Zellen von } A \text{ werden mit dem gleichen} \\ &\text{ Zeichen kodiert.}\\\pause
&\ (a)\Rightarrow A \text{ kann ignoriert werden}\\ \pause
&\ (b)\Rightarrow A \text{ wird konstant und } C \text{ bleibt Blank}\\ \pause
& \Rightarrow \sqrt{2} \text{ kann nicht dargestellt werden} \\
\text{Fall 2.2:} &\ analog.\pause
\end{align*}
$\Rightarrow$ rechtsseitig unendliches Band ist ausreichend.
\end{frame}

\begin{frame}{Beweis (3)}
\begin{block}{Reflection number der TM $R$} \pause
\begin{enumerate}
\item Setze Lesekopf auf die erste Zelle von $C$ \pause
\item Sobald der Lesekopf auf $B$ wandert, setzte ihn auf die erste Stelle von $C$ \pause
\item[$\rightarrow$] $R:=$ Anzahl der Wiederholungen der Schritte 1. und 2. \pause
\item[$\rightarrow$] $R \in \{1,2,...,\infty\}$
\end{enumerate}
\end{block}
\end{frame}

\begin{frame}{Beweis (4)}
\begin{block}{Reflection number der $\sqrt{2}$-Beschreibung $S$} \pause
\begin{enumerate}
\item Setze Lesekopf auf die erste Zelle von $B$ \pause
\item Sobald der Lesekopf auf $C$ wandert, setzte ihn auf die letzte Stelle von $B$ \pause
\item[$\rightarrow$] $S:=$ Anzahl der Wiederholungen der Schritte 1. und 2. \pause
\item[$\rightarrow$] $S \in \{1,2,...,\infty\}$ 
\end{enumerate}
\end{block}
\end{frame}

\begin{frame}{Beweis (5)}
\begin{align*}
\underline{\text{1.Fall}}&:  S \text{ ist endlich und } R > S \\ \pause
& \Rightarrow M \text{ loopt nach endlich vielen Schritten nur noch über } C\\ \pause
& \Rightarrow \text{nur ein endlicher Teil von } C \text{ wurde geändert} \\ \pause
\underline{\text{2.Fall}}&: S \text{ und } R \text{ sind beide unendlich} \\ \pause
& \Rightarrow \text{Lesekopf wird unbegrenzt oft auf } B \text{ aus } C \text{ zurückkehren:} \\ \pause
& \Rightarrow (a) \text{ Exkursionen in } C \text{ ist begrenzt } (b) \text{ oder unbegrenzt.}\\ \pause
& (a) \Rightarrow \text{ nur ein endlicher Teil von C wurde geändert}\\ \pause
& (b) \Rightarrow \text{ alles bis auf einen endlichen Teil des Bandes ändert sich dauerhaft} \qed
\end{align*}
\end{frame}

\begin{frame}
        \frametitle{Quellen}
         % nur angeben, wenn auch nicht im Text zitierte Quellen 
           % erscheinen sollen
\bibliographystyle{acm}
        \bibliography{sources}
\end{frame}



\end{document}



%========================
%
% END
%
%========================
