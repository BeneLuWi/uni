\chapter*{Zusammenfassung}
%% ==============================
In dieser Arbeit wurde die Verwendung nichtlinearer Taylormodelle in der Praxis anhand der zweidimensionalen H\e non-Abbildung untersucht. Die Anwendung spezieller Operationen auf den Taylormodellen und die Verwendung der reellen Zahlen der \verb+iRRAM+-Softwarebibliothek ergeben eine vielzahl an Konfigurationsmöglichkeiten, welche experimentell verglichen und evaluiert wurden, um eine möglichst optimale Belegung der Parameter zu finden.

Der Quellcode des im Zuge der Arbeit entstandenen Programms \verb+HOTM+ (\textit{High Order Taylor Model}) ist Online\footnote{\url{https://gitlab.rlp.net/s4beluek/hotaylormodels} (Stand 30.03.21)} und auf einem der Arbeit beigelegten Datenträger verfügbar.
