%% eval.tex
%% $Id: eval.tex 61 2012-05-03 13:58:03Z bless $

\chapter{Evaluation}
\label{ch:Evaluierung}
%% ==============================

\section{Konfiguration}
Die Implementierung nichtlinearer Taylormodelle ergibt eine Vielzahl von Konfigurationsmöglichkeiten und damit einen kaum abzudeckenden Suchraum nach der optimalen Einstellung:
\begin{enumerate}
 \item Grenzwert für Cleaning, Splitting
 \item Grad der Reduktion durch Sweeping
 \item Anzahl der zu erhaltenden Fehlersymbole
 \item Strategie beim Sweeping
 \item Heuristik für die Reihenfolge, in der Fehlersymbole gesweept werden
 \item Vorgehen beim Splitting
 \item Definition des initialen Taylormodells
\end{enumerate}
Diese Konfigurationen werden in \verb+hotm+ als JSON-Datei mit Hilfe einer JSON-Bibliothek\footnote{\url{https://github.com/nlohmann/json} Stand (Dezember 2020)} eingelesen und verarbeitet, um wiederholte Durchläufe mit leicht veränderten Parametern oder Batch-Runs zu vereinfachen. Listing \ref{list:config} zeigt ein Beispiel einer Konfigurationsdatei, mit der versuchsweise 1000 Iterationen der H\e non-Abbildung berechnet werden sollen. 





%%% Local Variables: 
%%% mode: latex
%%% TeX-master: "thesis"
%%% End: 
