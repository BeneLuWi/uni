%% grundlagen.tex
%% $Id: grundlagen.tex 61 2012-05-03 13:58:03Z bless $
%%

\chapter{Grundlagen}
\label{ch:Grundlagen}
%% ==============================
\section{H\e non-Abbildung}
Die H\e non-Abbildung bietet eine Möglichkeit, das Verhalten eines seltsamen Attraktors (\textit{engl. \anf{strange attractor}}), wie dem des Lorenz Systems, mit einer recht einfachen Abbildung zu untersuchen.

Konviergieren die Werte einer Funktion in einem bestimmten Wertebereich $R$, der Fangzone (\textit{engl. \anf{trapping zone}}), gegen einen Punkt oder eine Kurve, spricht man von einem Attraktor. Dieser Attraktor kann jedoch auch eine komplexere Struktur haben. Auf einem solchen seltsamen Attraktor springen die Werte hin und her und reagieren hochempfindlich auf Änderungen der Initialbedingungen. Dieses Verhalten lässt mit der Abbildung $x_{i+1}=y_i + 1- a x_i^2, y_{i+1} = bx_i$, beziehungsweise 
$$f(x,y) = (y + 1- a x^2,bx)$$
beobachten. $f$ erfüllt dieselben Kriterien, wie der durch eine dreidimensionale Differenzialgleichung entstehenden Lorenz-Attraktor, allerdings wurde $f$ so definiert, dass auch höhere Iterationszahlen leichter zu berechnen und zu analysieren sein sollen. Die H\e non-Abbildung bildet den $\mathbb{R}^2$ auf sich selbst ab. Dieser Vorgang besteht aus drei Schritten. Man betrachte eine Fläche entlang der $x$-Achse gelegen:

\textbf{Dehnen und Falten}
$$T': x'=x, y'= 1+ y - ax^2$$
Mit dem Parameter $a$ kann die Stärke der Biegung gesteuert werden.


\textbf{Kontrahieren}
$$T'': x''= b\cdot x, y''= y'$$
Ein $|b|<1$ bedeutet, dass sich die Fläche zusammen zieht. Wird $b$ zu groß gewählt, so entsteht eine zu große Kontraktion und der Attraktor ist schwerer erkennbar. Ist $b$ zu klein, ist der Effekt zu gering und das Verhalten der Abbildung ist nicht mehr chaotisch.


\textbf{Rotieren}
$$T''': x'''= y'', y'''= x''$$
Im letzten Schritt werden die Achsen vertauscht und somit dir Fläche um rotiert.


Die entstehende Abbildung hat unter Anderem folgende Eigenschaften:
\begin{itemize}
 \item Invertierbar: $(x_{n+1}, y_{n+1})$ kann eindeutig auf $(x_n, y_n)$ zurückgeführt werden.
 \item Kontrahiert Flächen: Mit $|b|<1$ werden Flächen kleiner.
 \item Besitzt eine Fangzone, die einen Attraktor enthält. Allerdings landen nicht immer alle Orbits in der Fangzone, da wegen $x^2$ Terme bestimmter Größe nach $\infty$ laufen können.
\end{itemize}


Abbildung \ref{fig:henonevo} zeigt die einzelnen Schritte einer Evolution der H\e non-Abbildung anhand eines Rechtecks, welches als nichtlineares Taylormodell definiert wurde.
$$x_0 = 0 + 1 \cdot \lambda_1 \hspace{.5cm} (\lambda_1 \in [0 \pm 0.4])$$
$$\ y_0 = 0 + 1 \cdot \lambda_2 \hspace{.5cm} (\lambda_2 \in [0 \pm 0.05])$$
Das Rechteck wird gedehnt und gefaltet, dann kontrahiert und zuletzt rotiert, beziehungsweise gespiegelt. 

\Abbildungps{tbh}{.7}{img/henon_evo.pdf}{fig:henonevo}{H\e non-Abbildung: Einfache Evolution}{Einfache Evolution der H\e non Abbildung, aufgeteilt in das initiale Rechteck und die drei Zwischenschritte.}

Für die Parameter $a=1.4$ und $b=0.3$ ergibt sich eine Fangzone in der sich die Funktionswerte auf einem seltsamen Attraktor bewegen, während die außerhalb gelegenen Punkte gegen unendlich laufen. Das bedeutet, dass ein Punkt, in der Fangzone, beziehungsweise auf dem Attraktor liegt, wiederum auf diesen abgebildet wird. Der Attraktor ist in Abbildung \ref{fig:strangeattractor} zu sehen. Hier wurden, ausgehend vom Punkt $(0,0)$, 10000 Iterationen der H\e non-Abbildung berechnet und jeweils das Ergebnis eingezeichnet. Es ist deutlich erkennbar, dass sich der Attraktor teils nahe am Rande der Fangzone bewegt. Bereits bei einer leichten Überschätzung des Ergebnisses kann dies dazu führen, dass die Funktionswerte die Fangzone verlassen, auch wenn der tatsächliche Wert eigentlich in dieselbe abgebildet würde. Dies kommt zum Tragen, wenn Intervalle, als Zahlendarstellung gewählt werden, wie es bei \verb+hotm+ der Fall ist, da dieser immer auch einen Bereich um den Wert herum abdecken. Liegt dieser nun außerhalb der Fangzone, wächst der Bereich durch die Quadrierung in jeder Iteration exponentiell und lässt somit keine aussagekräftigen Informationen ermitteln.


\Abbildungps{tbh}{.7}{img/attractor.pdf}{fig:strangeattractor}{H\e non-Abbildung: Attraktor}{Seltsamer Attraktor der H\e non-Abbildung für $a=1,4$ und $b=0.3$ mit 10000 Punkten.}

% Mit fortlaufenden Interationen wird das Rechteck immer weiter verzerrt. In Abbildung \ref{fig:henonevocolored} ist der Farbkodierung folgend, der Ursprung der Regionen im Initalrechteck erkennbar.

% \Abbildungps{tbh}{.7}{img/henon_evo_colored.pdf}{fig:henonevocolored}{H\e non-Abbildung: Einfache Evolution mit Farbkodierung}{Einfache Evolution der H\e non Abbildung, aufgeteilt in das initiale Rechteck und die drei Zwischenschritte mit Farbkodierung zur Rückführung der Regionen.}     


% \Abbildungps{tbh}{.9}{img/7iter_w_sweep.pdf}{fig:7iter}{H\e non-Abbildung: Mehrere Iterationen mit Farbkodierung}{Mehrere Iterationen mit Farbkodierung}

\section{Taylormodell}
Ein Taylormodell im Sinne der Erweiterung von \cite{DBLP:conf/macis/BrausseKM15} des Grundmodells von \cite{makino2001} besteht aus einem Polynom $p$ mit $k\in \mathbb{N}$ Variablen, geschlossenen Intervallen als Koeffizienten. Darin enthalten ist immer einem Monom $c_0$ des Grades 0, der das Kernintervall (\textit{Kernel}) des Taylormodells darstellt. Eine Variable oder ein Fehlersymbol $\lambda_i$ aus dem Vektor $\lambda = (\lambda_1, \dots, \lambda_k)$ steht für einen Wert aus dem dazugehörigen Supportintervall aus $S=(s_1, \dots, s_k)$ mit $\lambda_i \in s_i$ und wird dazu verwendet, unbekannte Werte, Rechenungenauigkeiten und funktionale Abhängigkeiten innerhalb eines oder zwischen mehreren Taylormodellen abzubilden. Die verwendeten Intervalle $c_n = [\tilde{c}_n \pm \varepsilon_n] \subseteq \mathbb{R}$ haben in \verb+hotm+ reelle Endpunkte und stellen mit $c'_n = [\tilde{c}_n \pm 0]$ auch Punktintervalle dar. Mit einem so definierten Taylormodell $T=\Sigma_n c_n \lambda^n$ kann exakte reelle Arithmetik betrieben werden, indem Rundungsfehler und Ungenauigkeiten, die beim Rechnen mit endlicher Genauigkeit entstehen können als Intervalle in den Fehlersymbolen berücksichtigt werden. Dies ist zwar auch mit einfacher Intervallarithmetik möglich, jedoch leidet die Präzision des Ergebnisses einer solchen Berechnung stark unter der schnell wachsenden Überschätzung, die sich aus der Tendenz von Intervallen ergibt, bei jeder Rechenoperation zu wachsen.

Des Weiteren können mit Taylormodellen im in dieser Arbeit behandelten zweidimensionalen Fall auch komplexere Flächen als achsenparallele Rechtecke mit Hilfe von Intervallen beschrieben werden.


In \cite{DBLP:conf/macis/BrausseKM15} werden drei Unterfamilien von Taylormodellen identifiziert, die sich in der Definition des Polynoms und dessen Koeffizienten unterschieden:
\begin{enumerate}
 \item Affine Arithmetik: Polynome des Grades $\leq 1$ mit Punktintervallen, außer beim Kernel.
 \item Generalisierte Intervallarithmetik: Polynome des Grades $\leq 1$ mit beliebigen Intervallen bei den Koeffizienten. \label{tm2}
 \item Klassische Taylormodelle: Polynome beliebigen Grades mit Punktintervallen, außer beim Kernel. \label{tm3}
\end{enumerate}

Das in dieser Arbeit verwendete Taylormodell ist eine Kombination aus \ref{tm2} und \ref{tm3}, und besteht aus Polynomen beliebiger Ordnung mit beliebigen Intervallen als Koeffizienten. Dadurch kann mit zwei solchen \textit{nichtlinearen Taylormodellen} im Zweidimensionalen komplexere Strukturen, wie Kurven höherer Ordnung beschrieben werden.

Abbildung \ref{fig:lin_vs_nonlin_1} zeigt lineare und eine nichtlineare Taylormodelle für
\begin{align*}
x = 0 + 1 \cdot \lambda_1 & \hspace{0.5cm} (\lambda_1 \in [0 \pm 0.4]) \\
 y = 0 + 1 \cdot \lambda_2 & \hspace{0.5cm} (\lambda_2 \in [0 \pm 0.1])
\end{align*}
und deren Abbildungen durch $f(x,y) = (y + 1- 1.4 x^2,0.3x)$ mit einer Farbkodierung, die den Ursprung der abgebildeten Flächen indiziert.

\Abbildungps{tbh}{.7}{img/lin_vs_nonlin_1.pdf}{fig:lin_vs_nonlin_1}{Linear und nichtlineare Taylormodelle: Vergleich}{Darstellung nichtlinearer und linearer Taylormodelle, die ein Rechteck beschreiben und deren Abbildung durch die Funktion $f(x,y) = (y + 1- 1.4 x^2,0.3x)$. Die Farbkodierung indiziert den Ursprung der abgebilteden Flächen.}

Sowohl die linearen, als auch die nichlinearen Taylormodelle umschließen den korrekten Bereich der Funktionswerte, allerdings ist die abgebildete Fläche der Nichtlinearen näher an der Fläche, die sich ergäbe, betrachtete man das Rechteck als Menge Punkten und bildete sie einzeln ab. Es ergibt sich eine geringere Überschätzung, aber auch ein komplexeres Polynom.


    \subsection{Arithmetische Operationen auf Taylormodellen}

Arithmetische Operationen auf Taylormodellen mit Intervallkoeffizienten bedeuten das Verrechnen von Polynomen, die wiederum Polynome ergeben. Für die Addition, Subtraktion und Multiplikation werden lediglich die entsprechenden Operationen auf die Polynome angewandt. Die Division erfordert



    \subsection{Spezielle Operationen auf Taylormodellen}
Um den durch Berechnungen wachsenden Grad des Polynomes und die Breite der Intervallkoeffizienten zu kontrollieren, wird in \cite{DBLP:conf/macis/BrausseKM15} \textit{Sweeping} und \textit{Splitting}, also Fegen und Teilen, vorgestellt. 

Sweeping reduziert den Grad eines Monoms $c_n \lambda_i^k$ mit $\lambda_i \in s_i$, indem eines Fehlersymbole durch das entsprechende Intervall aus $S$ ersetzt wird: 
\begin{align*}
 c_n \lambda_i^k \rightarrow c_n s_i \lambda_i^{k-1}
\end{align*}
Dies hat natürlich zur Folge, dass die Breite der Koeffizienten wächst und damit die Überschätzung der tatsächlichen Werte. Durch Splitting wird ein Monom in zwei Monome mit Punktintervallen zerteilt und ein neues Fehlersymbol eingeführt, das der Breite des Koeffizienten entspricht:
\begin{align*}
 [\tilde{c}_n \pm \varepsilon_n]\ \rightarrow [\tilde{c}_n \pm 0] + [\varepsilon_n \pm 0]\cdot  \lambda_n \hspace{0.5cm}, \lambda_n \in [0 \pm 1]
\end{align*}
Es ergibt sich der gegenteilige Effekt des Sweepings, da die Intervalle kleiner werden, der Grad des Polynoms jedoch erhöht wird.







\section{Mehrwertige Funktionen}













%%% Local Variables: 
%%% mode: latex
%%% TeX-master: "thesis"
%%% End: 
 
