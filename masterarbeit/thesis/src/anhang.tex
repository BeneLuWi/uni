\chapter{Anhang}

\section*{Algorithmen für Geobuckets}
\begin{algorithm}
\SetAlgoLined
\SetKwInOut{Input}{Eingabe}
\SetKwInOut{Output}{Ausgabe}
\Input{Polynome $p, q$ mit Monomen $m_{p_i}$ und $m_{q_j}$, wobei $i = \#p$ und $j = \#q$}
\Output{Polynom $p'$ mit geordneten Monomen}
$n \gets \#p, k\gets \#q$ \\
$buckets$ mit Platz geom. wachsender Größe anlegen: $\{2k, 4k, ..., 2^{\lceil log_2(n)\rceil -1 }k\}$\\
$i\gets 0 $\\
\While(\Comment*[f]{Bis jedes Monom aus $q$ betrachtet wurde}){$i < n$}{
    $p' \gets m_{p_i} \cdot q$\\
    \uIf{$i < (n-1)$}{
        $p' \gets p' + m_{p_{i+1}} \cdot q$\\
    }
    $i\gets i+2$\\
   $j\gets 0 $\\ 
    \While(\Comment*[f]{Sammle alle Buckets ein und speichere in $p'$}){$buckets[j]\neq \emptyset$}{
        $p' \gets p' + buckets[j]$\Comment*{Polynome gleicher Größe werden addiert}
        $buckets[j] \gets [\ ]$\\
        $j\gets j+1 $\\ 
    }
    \uIf(\Comment*[f]{Wurde $q$ nicht komplett betrachtet}){$i < n$}{
        $buckets[j] \gets p'$\Comment*{Speichere $p'$ im aktuell leeren Bucket $j$}
        $p' \gets 0$ \Comment*{Leere $p'$ für eine neue Iteration vorbereiten}
    }
    
}
\ForEach(\Comment*[f]{Merge aller Buckets ins Polynom}){$bucket \in buckets$}{
    $p' \gets p' + bucket$\Comment*{vom kleinsten zum größten}
}


\Return $p'$

 \caption{Polynommultipliaktion mit Geobuckets}
 \label{algo:mult}
\end{algorithm}


\begin{algorithm}[H]
\SetAlgoLined
\label{algo:add}
\SetKwInOut{Input}{Eingabe}
\SetKwInOut{Output}{Ausgabe}
\Input{Polynome $p, q$ mit Monomen $m_{p_i}$ und $m_{q_j}$, wobei $i = \#p$ und $j = \#q$}
\Output{Polynom $p'$ mit geordneten Monomen}
$i \gets 0, j \gets 0$ \\
\While(\Comment*[f]{Solange ein Polynom nicht komplett betrachtet wurde}){$i < \#p$ \textbf{and} $j  < \#q$} {
    \uIf{$m_{p_i} \succ m_{q_j}$}{
        $p' \gets p' + m_{p_i}$\\
        $i \gets i+1 $\\
    }
    \uElseIf{$m_{q_j} \succ m_{p_i}$}{
        $p' \gets p' + m_{q_j}$\\
        $j \gets j+1 $\\
    }
    \uElse(\Comment*[f]{Beide Monome haben dieselben Variablen}){
        $p' \gets p' + (m_{q_j} + m_{p_i})$\\
        $i \gets i+1; j \gets j +1 $\\
    }
}
\If{$i < \#p$ \textbf{or} $j  < \#q$}{
Füge den Rest zu $p$ hinzu
}

\Return $p$

 \caption{Addition zweier geordneter Polynome}
\end{algorithm}

\section*{Details der Implementierung}
Die Konfiguraion in Listing \ref{list:config} erstellt zwei Taylormodelle mit $x = [1] + [1] \cdot \lambda_0$ und $y = [1] + [0.1] \cdot \lambda_1 $ und den Supportintervallen $S=([0\pm2^{-1000}], [0\pm2^{-1000}])$, für die 1000 Iterationen der H\e non-Abbildung berechnet werden.

\begin{lstlisting}[language=JSON, caption=Bespielkonfiguration,captionpos=b, label=list:config]
 [
  {
    "comment": "Error 1000",
    "rel": 1,
    "linear": 0,
    "keep": 2,
    "error_in_lambda": true,
    "strategy": "SQUARE_ONLY",
    "iter": 1000,
    "prec": 50,
    "error": 1000,
    "decimals": 20,
    "sweep_to": 2,
    "micro_thresh": 1,
    "macro_thresh": 1,
    "a": 1.4,
    "b": 0.3,
    "support_space": [
      {
        "point": false,
        "mid": "0",
        "rad": "e"
      },
      {
        "point": false,
        "mid": "0",
        "rad": "e"
      }
    ],
    "x": {
      "kernel": {
        "point": true,
        "mid": "1",
        "rad": "0"
      },
      "monomials": [
        {
          "point": true,
          "mid": "1",
          "rad": "0",
          "lambdas": [{
            "exp": 1,
            "index": 0
          }]
        }
      ]
    },
    "y": {
      "kernel": {
        "point": true,
        "mid": "0.1",
        "rad": "0"
      },
      "monomials": [
        {
          "point": true,
          "mid": "1",
          "rad": "0",
          "lambdas": [{
            "exp": 1,
            "index": 1
          }]
        }
      ]
    }
  }
]
\end{lstlisting}
\section*{Darstellungen von Taylormodellen}
\Abbildungps{tbh}{1}{img/overlap_space.pdf}{fig:overlap}{H\e non-Abbildung: Einfache Iteration bei gesamter Abdeckung der Fangzone}{Einfache Iteration der H\e non-Abbildung mit einem Rechteck, dass die gesamte Fangzone abdeckt. Die Farbkodierung zeigt den Ursprung einer Region. Außerhalb der Fangzone gelegene Regionen werden wiederum außerhalb abgebildet, während sich die restlichen auf der Trajektorie des seltsamen Attraktors überlagern.}

\Abbildungps{tbh}{.8}{img/7iter_w_sweep.pdf}{fig:7iter}{H\e non-Abbildung: Mehrere Iterationen mit Farbkodierung}{Mehrere Iterationen der H\e non-Abbildung mit Farbkodierung, die den Ursprung der jeweiligen Regionen darstellt. Das initiale Rechteck wird durch zwei Taylormodelle aufgespannt bei denen während der Iteration auf das Hinzufügen und Entfernen von Fehlersymbolen verzichtet wird, damit diese Abhängigkeitsinformationen erhalten bleiben. Dies sorgt allerdings auch dafür, dass die Überschätzung schnell zu einem exponentiellen Wachstum der Intervalle gegen $\infty$ führt.}

