%% Einleitung.tex
%% $Id: einleitung.tex 61 2012-05-03 13:58:03Z bless $
%%

\chapter{Einleitung}
\label{ch:Einleitung}
%% ==============================

Das Rechnen mit reellen Zahlen und Funktionen auf denselben bedeutet die Verarbeitung einer unendlichen Menge an Informationen, die in der computergestützten Praxis lediglich mit einer endlichen Approximation beschrieben werden können \cite{Brattka2008}. Eine solche Beschränkung hat zu Folge, dass arithmetische Operationen keine exakten Ergebnisse ohne Fehler liefern. Dieser wird in der Disziplin der \textit{Exakten Reellen Arithmetik} durch Intervalle abgeschätzt, die den tatsächlichen Wert mitsamt Rundungsfehler umschließen und so ein korrektes Ergebnis garantieren können. Eine entsprechende Implementierung bietet die \verb.C++.-Softwarebibliothek \verb+iRRAM+ \cite{Mller2009EnhancingIE}, die reelle Zahlen zu einer beliebigen Genauigkeit, statt der häufig verwendetet \textit{double}-Präzision, annähert.
Jedoch entsteht durch die so betriebene Intervallarithmetik das Problem der Überschätzung des Wertebereichts, zum Beispiel beim mehrfachen Auftreten derselben Variable in einem Term. Ein weit verbreiteter Lösungsansatz ist die Verwendung von Taylormodellen, mit denen Zahlen als höherdimensionale Polynome mit reellen Koeffizienten und einem Fehlerintervall dargestellt werden \cite{DBLP:conf/macis/BrausseKM15} \cite{makino2001}. Zudem speichern Variablen (\anf{Fehlersymbole}) die Rechenfehler und deren Abhängigkeitsinformationen. Eine leicht veränderte Version der Taylormodelle wird im \verb.C++.-Programm \verb+Tangentspace+\footnote{\url{http://informatik.uni-trier.de/~mueller/Research/Tangentspace/} (Stand 29.03.2021)} implementiert. Die Taylormodelle bestehen hier aus linearen Polynomen mit beliebigen Intervallkoeffizienten und rationalen Endpunkten. Die Taylomodelle werden verwendet, um lineare Schranken für nichtlineare Funktionen zu bestimmen. Eine weitere Implementierung von Brauße et al. \cite{DBLP:conf/macis/BrausseKM15} verwendet reelle Endpunkte für die Intervallkoeffizienten und die Intervalle der Fehlersymbole. Das im Rahmen dieser Arbeit enstandene \verb.C++.-Programm \verb+HOTM+ (\textit{High Order Taylor Model}) bildet eine Kombination aus diesen beiden Herangehensweisen, da Intervalle mit beliebigen reellen Endpunkten verwendet werden. 

\paragraph{Zielsetzung}

Ziel dieser Arbeit war die Untersuchung nichtlinearer Taylormodelle in der praktischen Anwendung. Hierzu wurde das \verb.C++.-Programm \verb+HOTM+ (\textit{High Order Taylor Model}), beziehungsweise der gleichnamige Zahlentyp implementiert. Die Taylormodelle bestehen aus Polynomen mit Intervallkoeffizienten, welche die reellen Zahlen der \verb+iRRAM+-Software-Bibilothek als Endpunkte verwenden. Mit den so definierten Taylormodellen und damit einer mehrstufigen Intervallarchitektur ergibt sich eine Vielzahl an Konfigurationsmöglichkeiten von der Breite der Endpunkte der Intervalle, bis hin zu abstrakteren Operationen auf den Taylormodellen. An Hand von Berechnungen der H\e non-Abblildung werden diese evaluiert und versucht, eine möglichst optimale Einstellung für die Parameter der \verb+HOTM+ im zweidimensionalen Raum ($\mathbb{R}^2$) zu finden.

\paragraph{Gliederung}

Als praktisch ausgelegt Arbeit liegt der Schwerpunkt bei der Implementierung und den damit erzeugten experimentellen Ergebnissen. Die Basis hierfür bilden die in Kapitel \ref{ch:Grundlagen} beschriebenen Grundlagen, welche sich in H\e non-Abbildung, Taylormodelle und Lyapunov Exponenten gliedern. Diese Überblicke sollen das Verständnis über Entscheidungen in der Implementierung, die Deutung der Grafiken und der Ergebnisse vereinfachen. In Kapitel \ref{ch:Implementierung} werden die implementierten Operationen und Algorithmen der \verb+HOTM+ beschrieben. Mit der Anwerndung bei der H\e non-Abbildung werden diese in Kapitel \ref{ch:Evaluierung} evaluiert und verschiedenene Konfigurationen verglichen. Während der Implementierung und Durchführung der Experimente sind weiterführende Ideen und Anwendungsmöglichkeiten für die \verb+HOTM+ entstanden, auf die im Ausblick in Kapitel \ref{ch:fazit} neben der Diskussion der Ergebnisse eingegangen wird.

\paragraph{Relevante Literatur}
Die für diese Arbeit relevante Literatur lässt sich in vier Kategorien aufteilen, vergleichbar mit den später eingeführten Abstraktionsebenen, die sich aus dem Aufbau der \verb+HOTM+ ergeben. Für die grundlegende Zahlendarstellung wir die Implementierung der reellen Zahlen der \verb+iRRAM+ von Müller verwendet. Funktionsweisen und mögliche Anpassungen der Software-Bibilothek können den Quellen \cite{Mller2009EnhancingIE} und  \cite{mueller2001} entnommen werden. Für eine theoretische Grundlage der hier praktisch angewendeten Berechenbaren Analysis wird auf ein Tutorial von Brattka et al. \cite{Brattka2008} verwiesen. Spandl untersucht in \cite{DBLP:spandl} mit Hilfe der \verb+iRRAM+ den Lyapunov Exponenten für dynamische Systeme und zieht dabei Intervallmethoden heran, was einen zentralen Teil dieser Arbeit darstellt. 

Da die \verb+HOTM+ auf der nächst höheren Ebene aus Polynomen mit Intervallkoeffizienten bestehen, wurde eine allgemeine Intervallarithmetik, wie im Standardwerk von Moore \cite{moore1979} beschrieben implementiert. Yan et al. führen in \cite{geobuckets} einen Algorithmus ein, der die Addition von Polynomen effizient gestaltet, welcher in \cite{geobucketsmulti} für die Multiplikation formuliert ist. Mit dieser Basis können die Taylormodelle, wie in \cite{makino2001} und \cite{DBLP:conf/macis/BrausseKM15} beschrieben, implementiert werden.

In \cite{DBLP:conf/macis/BrausseKM15} sind zudem weiterführende Operationen auf Taylormodellen zu finden. Diese werden unter anderem im Programm \verb+Tangentspace+, auf dem die \verb+HOTM+s aufbauen, verwendet und ermöglicht geometrische Lösungen für Konfliktgesteuerte Gleichungslöser. Die Funktionsweise der so implementierten Taylormodelle wird an Hand der H\e non-Abblildung untersucht. Hierzu existieren neben dem ursprünglichen Paper von M. H\e non \cite{henon1976}, viele Experimente und Untersuchungen, wie Oliviera et al., die sich in \cite{MartinsdeOliveira2020} mit verschiedenen Parameterbelegungen beschäftigen.






%%% Local Variables: 
%%% mode: latex
%%% TeX-master: "thesis"
%%% End: 
