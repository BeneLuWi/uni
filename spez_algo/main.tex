\documentclass[ngerman]{scrartcl}
\usepackage{amsmath,amsthm,amssymb}
\usepackage[T1]{fontenc}
\usepackage[utf8]{inputenc}
\usepackage{lmodern}
\usepackage{graphicx}

\usepackage{hyperref}

\title{Algorithmen und Datenstrukturen (Master) \\ WiSe 19/20}
\author{Benedikt Lüken-Winkels}
\begin{document}

\maketitle
\tableofcontents
\newpage


\subsection*{Wörterbuchproblem}
Menge S mit n Schlüssln aus einem Universum U.
Operationen: INSERT (darauf achten, dass die Balance nicht verloren geht), DELETE, LOOKUP (Im Baum runterlaufen, bis das Element gefunden wurde)
\paragraph{Situationen}
\begin{enumerate}
    \item U linear gehordnet, also existiert ein $ \leq $-Test $ \Rightarrow $ Suchbäume
    \item U ist ein Intervall $ \{0,..., N-1\} $ der gesamten Zahlen $ \Rightarrow $ Hashing
\end{enumerate}
\subsubsection*{zu 1}
\paragraph{Randomisierte Suchbäume}
Idee: Benutze Zufallszahlen zur Balancierung eines binären Suchbaums
\paragraph{Binärer Suchbaum (Knoten-Orientiert)}
Schlüssel werden in den n Knoten eines binären Baums gespeichert, sodass im linken Unterbaum des Knotens mit Schlüssel x alle Schlüssel $ < x $ \textbf{und} im rechten Unterbaum alle $ > x $. Balanciert $ \Rightarrow Höhe(T)\leq logn$.  Degeneriert $ \Rightarrow Höhe(T) = O(n)$





\end{document}










