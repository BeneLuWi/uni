%=====================================================
%=====================================================
%=====================================================
\section{Bellman/Ford - Allgemeiner Graph}
Geben Sie den Bellman/Ford Algorithmus an und analysieren Sie seine Laufzeit.
\par 
\begin{algorithm}
 \caption{Bellman/Ford}
  \SetKwInOut{Input}{input} 
 \Input{Graph $G(V,E)$ \newline
 Knoten $s$\newline
 Kostenfunktion $c$\newline
 Knotenarray $DIST$\newline
 Knotenarray $PRED$}
 
 Queue $Q$; \hspace{5cm} //Kandidatenmenge\\
 Nodearray $count \gets (G, 0)$ \\
 
 \For{$v\in G$}{
 $DIST[v] \gets \infty$\\ 
 $PRED[v] \gets NULL$\\
 }
 
 $DIST[s] \gets 0$\\
 $Q \gets Q \cup s$\\
 \While{$Q \neq \emptyset$}{
    $u \gets Q.pop()$\\    
    $count[u]++$\\
    \uIf{$count[u] > |V|$}{
        \Return negativer Zyklus gefunden\\
    }
    \For{$e \in u.out\_edges()$}{
        $v \gets e.target()$\\
        \uIf{$DIST[u] + c(e) < DIST[v]$} {
        $DIST[v] = DIST[u] + c(e)$\\
        $PRED[v] = e$\\
        $Q \gets Q\cup v$\\
        }
        
    }
 }
 \Return kein negativer Zyklus gefunden \\
\end{algorithm}

$PRED$ wird dazu verwendet, um nachher den kürzesten Weg zu einem Knoten nachverfolgen zu könnten. $PRED$-Verweise ändern sich daher, wenn sich der kürzeste Weg ändert.


\paragraph{Laufzeit} 
\[
 n \cdot (\underbrace{\text{Iterationen über alle Knoten} | \text{ausgehende Kanten}}_{O\underbrace{(\sum_{v\in V} (1 + outdeg(v)))}_{n+m}})
\]
$\Rightarrow O(n (n+m))$ als Gesamtlaufzeit. Ist der Graph zusammenhängend, so gilt $m \geq n -1$ und die Laufzeit wird von $m\cdot n$ dominiert $\Rightarrow O(nm)$



%=====================================================
%=====================================================
%=====================================================
\section{SSSP-Algorithmus ohne neagtive Zyklen - Dijkstra}
\label{sec-1}
Geben Sie einen SSSP-Algorithmus an, der O(n + m) als Laufzeit hat und auf nicht-negativen Netzwerken funktioniert.
\par 
\begin{algorithm}[H]
\caption{dist - verfeinert}
\For{$v\in G$}{
 $DIST[v] \gets \infty$\\ 
 $PRED[v] \gets NULL$\\
 }
DIST[s] $\gets 0$ \\
Priority Queue $PQ.insert(s,0)$ \\
\While{$PQ \neq \emptyset$}{
    wähle Knoten $u \in PQ$ mit $DIST[u]$ minimal: $PQ.del\_min()$;\\
    \For{$v\in V$ mit e = (u,v) $\in E$}{
        $d \gets DIST[u] + c(e)$\\
        \If{DIST[v] > d} {
        \uIf{$DIST[v] = \infty$}{
            $PQ.insert(v, d)$\\
        }
        \uElse{
            $PQ.decrease\_p(v, d)$\\
        }
        $DIST[v] \gets d$\\
        $PRED[v] \gets e$
        }
    }
}
\end{algorithm}
Priority = Distanz

\paragraph{Laufzeit} $O(n+m)$ durch perfekte Wahl.
\begin{itemize}
 \item Operationen auf dem Graphen: $O(n+m)$
 \item Priority Queue 
\end{itemize}
$\Rightarrow$ Gesamtlaufzeit $O(n \cdot (T_{insert}(n) + T_{delmin}(n)+ T_{empty}(n)) + m \cdot T_{decrease}(n) )$\\
$m$ ist der Flaschenhals des Algorithmus.
\paragraph{Realisierung der Datenstruktur}
\begin{itemize}
 \item binärer Min-Heap
 \item balancierter Baum
 \item[$\Rightarrow$] $O((n+m)\log n)$
\end{itemize}

Mit Fibonacci Heap: Insert $O(1)$, Delmin $(\log n)$) $ \Rightarrow O(n \log(n) + m)$ (Decrease $O(1)$, 

%=====================================================
%=====================================================
%=====================================================
\section{SSSP-Algorithmus azyklische Netzwerke}
Geben Sie einen SSSP-Algorithmus an, der O(n + m) als Laufzeit hat und auf azyklischen Graphen funktioniert.
\par
Azyklische Graphen besitzen eine topologische Sortierung.

\begin{algorithm}
 \caption{SSSP-Alogrithmus mit Topsort}
 \SetKwInOut{Input}{input} 
 Knotenarray $INDEG(G,0)$ \\
 \For{$v \ in V$}{
    $INDEG[v] \gets indeg(v)$ \\
    \If{$INDEG[v] = 0$} {
        $zero.append(v)$\\
        }
    $DIST[v] \gets \infty$

    }
    $DIST[s] \gets 0$\\
    \While{$zero$ is not empty}{
        $u \gets zero.pop()$\\
        \For{$(u,v) \in E$}{
            $INDEG[v] \gets INDEG[v] -1$\\
        
        \If{$INDEG[v] = 0$}{
            $zero.append(v)$\\
        }
        $d \gets DIST[u] + c((u,v))$\\
        \If{$DIST[v] > d$}{
            $DIST[v] \gets d$
        }
        }
    }
 
 
 
\end{algorithm}

\paragraph{Laufzeit} $O(n+m)$ durch perfekte Wahl mit topologischem Sortieren





%=====================================================
%=====================================================
%=====================================================
\section{SSSP-Algorithmus mit negativen Zyklen}
\label{sec-2}
Geben Sie einen SSSP-Algorithmus an, der negative Zyklen erkennt und ihn ausgibt.
Wie ist die Laufzeit?
\par
Bellman/Ford






\newpage

%=====================================================
%=====================================================
%=====================================================
\section{Maxflow Labeling}
\label{sec-3}
Geben Sie einen Labeling-Algorithmus zur Lösung des Maxflow-Problems in Pseudo-Code an und analysieren Sie die Laufzeit.

\paragraph{Bemerkung} Der Wert des maximalen Flusses ist gleich der Kapazität eines minimalen (s,t)-Schnittes

\paragraph{Laufzeit} In jeder Iteration von Augment erhöht der Algorithmus den Flusswert um min 1. Daher können maximal $F_{max} \leq n\cdot U$ Iterationen benötigt werden. ($U =$ maximale Kapazität). Eine Labeling Iteration kostet $O(n+m) \Rightarrow O(n^2 \cdot U + m \cdot n \cdot U)$.\\
Ist G zusammenhängend, so git $m \geq n -1 \Rightarrow$ \textbf{Insgesamt} $O(n\cdot m \cdot U)$

\underline{Worst Case:} U kann sehr groß sein, daher keine polynomielle Laufzeit in $m$ und $n$

\begin{algorithm}[!htb]
 \caption{Maxflow Labeling}
 \SetKwInOut{Input}{input}
 \Input{Graph $G = (V,E)$, Knoten $s,t$, Kapazitätsfunktion $u$, Flussfunktion $x$}

 Liste $S\gets [\ ]$\\
 Knotenarray $PRED(G, NULL)$\\
 Kantenarray $labelled(G)$\\
 \For{$v\in V$}{
    $labelled[v] \gets false$
 }
 \While{true}{
    $labelled[s] \gets true$\\
    $S.append(s)$\\
    \While{$!S.empty()$}{
        $v \gets S.pop()$\\
        
        \For(\tcp*[f]{Forwärtskanten}){$e =(v,w) \in E,$}{
            \If{$x(e) = u(e)$ or $labelled[w]$}{
                continue \tcp*[f]{Restkapazität ist = 0 oder schon besucht}
            }
            $labelled[w] \gets true$\\
            $PRED[w] \gets e$\\
            $S.append(w)$\\
        
        }
        \For(\tcp*[f]{Rückkanten}){$e =(w,v) \in E,$}{
            \If{$x(e) = 0$ or $labelled[w]$}{
                continue \tcp*[f]{Fluss ist = 0 oder schon besucht}
            }
            $labelled[w] \gets true$\\
            $PRED[w] \gets e$\\
            $S.append(w)$\\
        }
        \If{$labelled[t]$}{
            $S.clear()$\\
        }
        
    }
    \uIf{$labelled[t]$}{
        $AUGMENT(G, s, t, PRED, u, x)$\\
    }
    \Else{
        $break$\\ 
    }
 }
 
 
\end{algorithm}


\begin{algorithm}[!htb]
 \caption{Augment - Pfaderhöhung}
 \SetKwInOut{Input}{input}
 \Input{Graph $G = (V,E)$, Knoten $s,t$, Kapazitätsfunktion $u$, Flussfunktion $x$}
 $\delta \gets \infty$\tcp*[f]{Restkapazität des Pfades P}\\
 $v \gets t$\\
 \While{$v\neq s$}{
    $e \gets PRED[v]$\\
    \uIf(\tcp*[f]{Rückkante}){$e = (v,w)$}{
        $r \gets x(e)$\\
        $v \gets w$\\
    }
    \Else(\tcp*[f]{$e = (w,v)$ Forwärtskante }){
        $r \gets u(e) - x(e)$\\
        $v \gets w$\\
    }
    \If{$r < delta$} {
        $\delta \gets r$\tcp*[f]{neues minimum}
    }
    
}
$v \gets t$\\
\While{$v\neq s$}{
     $e \gets PRED[v]$\\
    \uIf(\tcp*[f]{Rückkante}){$e = (v,w)$}{
        $x(e) \gets x(e)- \delta$\\
        $v \gets w$\\
    }
    \Else(\tcp*[f]{$e = (w,v)$ Forwärtskante}){
        $x(e) \gets x(e) + \delta $\\
        $v \gets w$\\
    }
}
 
 
\end{algorithm}


\newpage
%=====================================================
%=====================================================
%=====================================================
\section{Capacity-Scaling}
\label{sec-4}
Geben Sie CAPACITY-SCALING unter Verwendungen des Maxflow Labeling Algorithmus an. Begründen Sie die Laufzeit.
\paragraph{Bemerkung} Maxflow mit möglichst hoher Restkapazität im Pfad P oder kürzesten erhöhenden Pfad. Letzte $\Delta$-Phase entspricht dem normalen Labeling Alogrithmus $\rightarrow$ Korrektheit.

\begin{algorithm}[!ht]
 \caption{Capacity-Scaling}
 \SetKwInOut{Input}{input}
 \Input{Graph $G = (V,E)$, Knoten $s,t$, Kapazitätsfunktion $u$, Flussfunktion $x$, Maximale Kapazität $U$}

 $\Delta \gets 2^{\log U}$\\
  \While{$\Delta > 1 $}{
    $MFLabeling(G, s, t, PRED, u, x)$\\
    $\Delta \gets \frac{\Delta}{2}$
  }

 
\end{algorithm}



\begin{algorithm}[!htb]
 \caption{Maxflow Labeling mit $\Delta$}
 \SetKwInOut{Input}{input}
 \Input{Graph $G = (V,E)$, Knoten $s,t$, Kapazitätsfunktion $u$, Flussfunktion $x$, Restkapazitätsgrenzwert $\Delta$}

 Liste $S\gets [\ ]$\\
 Knotenarray $PRED(G, NULL)$\\
 Kantenarray $labelled(G)$\\
 \For{$v\in V$}{
    $labelled[v] \gets false$
 }
 \While{true}{
    $labelled[s] \gets true$\\
    $S.append(s)$\\
    \While{$!S.empty()$}{
        $v \gets S.pop()$\\
        
        \For(\tcp*[f]{Forwärtskanten}){$e =(v,w) \in E,$}{
            \If{$u(e) - x(e) < \Delta $ or $labelled[w]$}{
                continue \tcp*[f]{Restkapazität ist $< \Delta$ oder schon besucht}
            }
            $labelled[w] \gets true$\\
            $PRED[w] \gets e$\\
            $S.append(w)$\\
        
        }
        \For(\tcp*[f]{Rückkanten}){$e =(w,v) \in E,$}{
            \If{$x(e) < \Delta$ or $labelled[w]$}{
                continue \tcp*[f]{Fluss ist $< \Delta$ oder schon besucht}
            }
            $labelled[w] \gets true$\\
            $PRED[w] \gets e$\\
            $S.append(w)$\\
        }
        \If{$labelled[t]$}{
            $S.clear()$\\
        }
        
    }
    \uIf{$labelled[t]$}{
        $AUGMENT(G, s, t, PRED, u, x)$\\
    }
    \Else{
        $break$\\ 
    }
 }
 
 
\end{algorithm}

\paragraph{Laufzeit} Jede $\Delta$-Phase führt maximal $2m$ Erhöhungen aus. Die Restkapazität eines (s,t)-Schnittes $r[S,\overline{S}] \leq m\cdot \Delta$, da jede Kante zwischen $S$ und $\overline{S}$ höchstens Restkapazität $\Delta$ hat. Die nächste $\frac{\Delta}{2}$-Phase führt dann maximal $\frac{m\Delta}{\Delta/2} \leq 2m$ Erhöhungen aus. Die Laufzeit einer $\Delta$-Phase ist daher $O(2m \cdot m) \Rightarrow O(m^2)$ 

Insgesamt ergibt sich die Laufzeit $O(m^2 \log U)$. Polynomiell, aber nicht streng polynomiell.

\newpage


%=====================================================
%=====================================================
%=====================================================
\section{Preflow Push}
\label{sec-5}
\paragraph{Admissible} Distanz-Funktion von $t\ d$ mit $d(t)= 0$. $d(i) \leq d(j)+ 1$ für alle Kanten $(i,j) \in G(x)$. Gilt für eine Kante $(i,j)$, dass $d(i) = d(j) + 1$, so ist sie $admissible$

\subsection{Generisch}
\paragraph{Laufzeit}
\begin{itemize}
 \item Relabel: $O(2n^2)$. $d(i)$ ist $< 2n$ und da jedes mal nur um 1 erhöht wird ergibt $n \cdot 2n$
 \item Saturierende Push-Operationen: $O(nm)$. Zwischen 2 SatPush-Operationen derselben Kante muss $d$ und 2 erhöht worden sein. Da das $d(i) < 2n$, geht das für jede Kante $n$ mal.
 \item Nicht-Saturierende Push-Operationen: $O(n^2m)$. Potential (Zustand des Netzwerks als numerischer Wert) $\Phi$ als Summe aller Distanz-Labels aktiver Knoten $i\in I$ (also $e(i)>0$)
 \begin{enumerate}
  \item Anfang: $\Phi \leq 2n^2$
  \item Bei Termination: $\Phi = 0$ und keine aktiven Knoten mehr.
 \end{enumerate}
 Nun werden 2 Fälle unterschieden:\\
 \underline{Fall 1:} keine admissible Edge für $i$: $d(i)$ wird um mindestens 1 und höchstens $2n$ erhöht. Eine Erhöhung von $2n \cdot n = 2n^2$ ist also möglich.
 
 
 \underline{Fall 2.1:} für die admissible Edge wird ein saturierendes Push ausgeführt: Die Anzahl der aktiven Knoten kann um 1 erhöht und das Potential um $2n$, also $2n \cdot nm$.
 
 
 \underline{Fall 2.2:} für die admissible Edge wird ein nicht-saturierendes Push ausgeführt. Bei einem nicht saturierenden Push wird $\Phi$ um mindestens 1 verringert, da falls ein neuer Knoten $j$ aktiviert wird $d(i) = d(j) + 1$ gilt ($i$ ist nun deaktiviert).
 
\end{itemize}
Das Potential kann also auf $2n^2 + 2n^2 + 2n^2m$ (Anfangswert + Fall 1 + Fall 2.1) wachsen. Die nicht-saturierenden Pushes verringern um min 1, also $O(n^2m)$ nicht saturierende Pushes.
Die Laufzeit wird durch durch die Anzahl der nicht-saturierenden Pushes dominiert $\Rightarrow O(n^2m)$. Die Kapazität spielt hier keine Rolle mehr für die Laufzeit.

\begin{algorithm}
 \caption{Generischer Preflow-Push Algorithmus}
 \For{$i\in V$}{
    $d(i)\gets 0$\\
    $e(i)\gets 0$\\
 }
 \For{$(i,j) \in E$}{
    $x_{ij}\gets 0$
 }
 \For(\tcp*[f]{Saturiere alle aus s ausgehenden Kanten}){$j\in V$ mit $(s,j) \in E$}{
    $x_{sj}\gets u_{sj}$\\
    $e(s)\gets e(s) - u_{sj}$\\
    $e(j)\gets e(j) + u_{sj}$
 }
 $d(s) \gets n$\tcp*[f]{Hebe $s$ auf auf Level $n$}\\
 \While{$\exists i \in V\ mit\ e(i) >0$}{
    Wähle $i$\\
    PUSH/RELABEL($i$)
 }
 
 
 
\end{algorithm}



\begin{algorithm}
 \caption{Push/Relabel}
 \uIf(\tcp*[f]{Push}){$i$ hat eine admissible Kante $(i,j)\in G(x)$}{
    Wähle Kante $(i,j)$\\
    $\delta \gets \min\{e(i), r_{ij}\}$\\
    $x_{ij} \gets x_{ij} + \delta$\\
    $e(i) \gets e(i) - \delta$\\
    $e(j) \gets e(j) + \delta$
 }
 \Else(\tcp*[f]{Relabel}){
    $d(i) \gets \min\{ d(j) | (i,j) \in G(x) \} + 1$
 }
 
 
 
\end{algorithm}



\subsection{FIFO Preflow-Push}
Aktive Knoten sind in einer Queue. Folge von Operationen auf einem Knoten sind möglich.
\paragraph{Laufzeit} Der Ablauf wird in Phasen eingeteilt. $i$-te Phase ist die Behandlung aller Knoten, die sich nach Phase $i-1$ in $Q$ befinden. Anzahl der Phasen ist $\leq 2n^2$. Pro Phase wird jeder Knoten maximal einmal behandelt und führt für jeden Knoten maximal ein nicht-saturierendes Push. $\Rightarrow n\cdot 2n^2 = O(n^3)$


\begin{algorithm}
 \caption{FIFO-Queue Preflow-Push Algorithmus}
 $Q\gets empty\ Queue$\tcp*[f]{Alle aktiven Knoten mit $e> 0$}\\
 \For{$i\in V$}{
    $d(i)\gets 0$\\
    $e(i)\gets 0$\\
 }
 \For{$(i,j) \in E$}{
    $x_{ij}\gets 0$
 }
 \For(\tcp*[f]{Saturiere alle aus s ausgehenden Kanten}){$j\in V$ mit $(s,j) \in E$}{
    $x_{sj}\gets u_{sj}$\\
    $e(s)\gets e(s) - u_{sj}$\\
    $e(j)\gets e(j) + u_{sj}$\\
    $Q.append(j)$\\
 }
 $d(s) \gets n$\tcp*[f]{Hebe $s$ auf auf Level $n$}\\
 
 
 
 \While{Q is not empty}{
    $i \gets Q.pop()$\\
    \For{$(i,j) \in G(x)$}{
        \If(\tcp*[f]{Wähle eine admissible Kante}){d(i) = d(j) + 1} {
            $\delta \gets \min\{e(i), r_{ij}\}$\tcp*[f]{Push}\\
            $x_{ij} \gets x_{ij} + \delta$\\
            $e(i) \gets e(i) - \delta$\\
            $e(j) \gets e(j) + \delta$\\
            \If{$e(j)\neq 0$}{
            $Q.append(j)$
            }
        }
        \If{e(i) = 0} {
        \textbf{break}
        }
    }
    \If(\tcp*[f]{Keine admissible Kante verfügbar}){$e(i) > 0$} {
        $d(i) \gets \min\{ d(j) | (i,j) \in G(x) \} + 1$\tcp*[f]{Relabel}\\
        $Q.append(i)$
    }
    
 }
 
 
\end{algorithm}

\newpage

\subsection{Highest-Label Preflow-Push}
Wähle einen Knoten mit maximalen Dist-Label in Priority-Queue. Da sich die Dist-Labels auf das Intervall $\{0,\dots, 2n \}$ beschränken sind Buckets möglich.
\paragraph{Realisierung der Priority Queue} Distanzen sind aus einem beschränkten Intervall $\{0, \dots,  2n\}\Rightarrow$ Feld von Listen (Bucket-Array). Pointer auf das maximale Bucket
\paragraph{Laufzeit} 
\begin{itemize}
 \item Q.insert: füge i in das entsprechende Bucket und überprüfe max-Pointer $\Rightarrow O(1)$
 \item Q.delmax: Entferne vorderstes Element eines Buckets und aktualisiere max-Pointer $\Rightarrow O(n)$
\end{itemize}

$O(n^2 \sqrt{m})$


\begin{algorithm}
 \caption{Priority Queue Preflow-Push (Highest Label)}
 $Q\gets empty\ Queue$\tcp*[f]{Alle aktiven Knoten mit $e> 0$}\\
 \For{$i\in V$}{
    $d(i)\gets 0$\\
    $e(i)\gets 0$\\
 }
 \For{$(i,j) \in E$}{
    $x_{ij}\gets 0$

 }
 \For(\tcp*[f]{Saturiere alle aus s ausgehenden Kanten}){$j\in V$ mit $(s,j) \in E$}{
    $x_{sj}\gets u_{sj}$\\
    $e(s)\gets e(s) - u_{sj}$\\
    $e(j)\gets e(j) + u_{sj}$\\
    $Q.insert(j, d(j))$
 }
 $d(s) \gets n$\tcp*[f]{Hebe $s$ auf auf Level $n$}\\
 
 
 
 \While{Q is not empty}{
    $i \gets Q.delmax()$\\
    \For{$(i,j) \in G(x)$}{
        \If(\tcp*[f]{Wähle eine admissible Kante}){d(i) = d(j) + 1} {
            $\delta \gets \min\{e(i), r_{ij}\}$\tcp*[f]{Push}\\
            $x_{ij} \gets x_{ij} + \delta$\\
            $e(i) \gets e(i) - \delta$\\
            $e(j) \gets e(j) + \delta$\\
            \If{$e(j)\neq 0$}{
                $Q.insert(j,d(j))$
            }
        }
        \If{e(i) = 0} {
        \textbf{break}
        }
    }
    \If(\tcp*[f]{Keine admissible Kante verfügbar}){$e(i) > 0$} {
        $d(i) \gets \min\{ d(j) | (i,j) \in G(x) \} + 1$\tcp*[f]{Relabel}\\
        $Q.insert(i,d(i))$
    }
 }
 
\end{algorithm}






\newpage
%=====================================================
%=====================================================
%=====================================================
\section{Min-Cost-Flow-Problem (MCF)}
Wie lautet die reduzierte Kosten-Optimalität für das Min-Cost-Flow-Problem? Folgern Sie aus ihrer Gültigkeit, dass im Restnetzwerk keine negativen Zyklen existieren.
\paragraph{Eingaben} Kostenfunktion $c$, Supplyfunktion $b$, Flussfunktion $x$, Kapazitätsfunktion $u$
\paragraph{Feasible} Ein Fluss $x$, für den alle Kanten  Kapazitätsbedingung und  Massebalance gelten, ist feasible.

\paragraph{Optimalitätsbedingung (Negative Cycle Optimality)} Wenn im Restnetzwerk $G(x)$ kein negativer Zyklus enthalten ist.

\paragraph{Reduzierte Kosten} Potentialfunktion $\pi$. $c_{ij}^{\pi} = c_{ij} + \underbrace{\pi(j) - \pi(i)}_{\Delta \pi}$

\paragraph{Reduzierte Kostenoptimalität} $x$ optimal $\Leftrightarrow \exists \text{ Potential } \pi: \forall (i,j) \in G(x) : c_{ij}^{\pi} \geq 0$ also alle Kosten mit addiertem Potential (zB negative Distanz)


\paragraph{Complementary Slackness Optimality} arbeitet auf G.

\paragraph{Laufzeit} $U = \max u_{ij}$, $C = \max |c_{ij}|$. 
\begin{itemize}
 \item Pro Iteration wird um mindestens 1 reduziert. 
 \item Die Spanne der Kosten geht von $-mCU$ bis $mCU$, also $\leq mCU$ Iterationen
 \item Eine Iteration kostet $O(nm)$ (Bellman/Ford)
\end{itemize}
$\Rightarrow O(nm^2 CU )$



\begin{algorithm}
 \caption{Negative Cycle Canceling - Eleminierung von negativen Zyklen}
  Erstelle einen Maxflow in G\\
 $W \gets Bellman/Ford/negCycle(G)$ \\
 \While{$W \neq NULL$}{
    $\delta \gets \min\{r_{ij}: (i,j)\in W \}$\\
    $augment(W, \delta)$
  
  }
 
\end{algorithm}













