%=====================================================
%=====================================================
%=====================================================
\section{Bellman/Ford - Allgemeiner Graph}
Geben Sie den Bellman/Ford Algorithmus an und analysieren Sie seine Laufzeit.
\par 
\begin{algorithm}
 \caption{Bellman/Ford}
  \SetKwInOut{Input}{input} 
 \Input{Graph $G(V,E)$ \newline
 Knoten $s$\newline
 Kostenfunktion $c$\newline
 Knotenarray $DIST$\newline
 Knotenarray $PRED$}
 
 Queue $Q$; \hspace{5cm} //Kandidatenmenge\\
 Nodearray $count \gets (G, 0)$ \\
 
 \For{$v\in G$}{
 $DIST[v] \gets \infty$\\ 
 $PRED[v] \gets NULL$\\
 }
 
 $DIST[s] \gets 0$\\
 $Q \gets Q \cup s$\\
 \While{$Q \neq \emptyset$}{
    $u \gets Q.pop()$\\    
    $count[u]++$\\
    \uIf{$count[u] > |V|$}{
        \Return negativer Zyklus gefunden\\
    }
    \For{$e \in u.out\_edges()$}{
        $v \gets e.target()$\\
        \uIf{$DIST[u] + c(e) < DIST[v]$} {
        $DIST[v] = DIST[u] + c(e)$\\
        $PRED[v] = e$\\
        $Q \gets Q\cup v$\\
        }
        
    }
 }
 \Return kein negativer Zyklus gefunden \\
\end{algorithm}

$PRED$ wird dazu verwendet, um nachher den kürzesten Weg zu einem Knoten nachverfolgen zu könnten. $PRED$-Verweise ändern sich daher, wenn sich der kürzeste Weg ändert.


\paragraph{Laufzeit} 
\[
 n \cdot (\underbrace{\text{Iterationen über alle Knoten} | \text{ausgehende Kanten}}_{O\underbrace{(\sum_{v\in V} (1 + outdeg(v)))}_{n+m}})
\]
$\Rightarrow O(n (n+m))$ als Gesamtlaufzeit. Ist der Graph zusammenhängend, so gilt $m \geq n -1$ und die Laufzeit wird von $m\cdot n$ dominiert $\Rightarrow O(nm)$



%=====================================================
%=====================================================
%=====================================================
\section{SSSP-Algorithmus ohne neagtive Zyklen - Dijkstra}
\label{sec-1}
Geben Sie einen SSSP-Algorithmus an, der O(n + m) als Laufzeit hat und auf nicht-negativen Netzwerken funktioniert.
\par 
\begin{algorithm}[H]
\caption{dist - verfeinert}
\For{$v\in G$}{
 $DIST[v] \gets \infty$\\ 
 $PRED[v] \gets NULL$\\
 }
DIST[s] $\gets 0$ \\
Priority Queue $PQ.insert(s,0)$ \\
\While{$PQ \neq \emptyset$}{
    wähle Knoten $u \in PQ$ mit $DIST[u]$ minimal: $PQ.del\_min()$;\\
    \For{$v\in V$ mit e = (u,v) $\in E$}{
        $d \gets DIST[u] + c(e)$\\
        \If{DIST[v] > d} {
        \uIf{$DIST[v] = \infty$}{
            $PQ.insert(v, d)$\\
        }
        \uElse{
            $PQ.decrease\_p(v, d)$\\
        }
        $DIST[v] \gets d$\\
        $PRED[v] \gets e$
        }
    }
}
\end{algorithm}
Priority = Distanz

\paragraph{Laufzeit} $O(n+m)$ durch perfekte Wahl.
\begin{itemize}
 \item Operationen auf dem Graphen: $O(n+m)$
 \item Priority Queue 
\end{itemize}
$\Rightarrow$ Gesamtlaufzeit $O(n \cdot (T_{insert}(n) + T_{delmin}(n)+ T_{empty}(n)) + m \cdot T_{decrease}(n) )$\\
$m$ ist der Flaschenhals des Algorithmus.
\paragraph{Realisierung der Datenstruktur}
\begin{itemize}
 \item binärer Min-Heap
 \item balancierter Baum
 \item[$\Rightarrow$] $O((n+m)\log n)$
\end{itemize}

Mit Fibonacci Heap: Insert $O(1)$, Delmin $(\log n)$) $ \Rightarrow O(n \log(n) + m)$ (Decrease $O(1)$, 

%=====================================================
%=====================================================
%=====================================================
\section{SSSP-Algorithmus azyklische Netzwerke}
Geben Sie einen SSSP-Algorithmus an, der O(n + m) als Laufzeit hat und auf azyklischen Graphen funktioniert.
\par
Azyklische Graphen besitzen eine topologische Sortierung.

\begin{algorithm}
 \caption{SSSP-Alogrithmus mit Topsort}
 \SetKwInOut{Input}{input}
 \Input{Graph $G$, Knoten $s$, Kostenfunktion $c$, Knotenarray $DIST$}
 Knotenarray $INDEG(G,0)$ \\
 
\end{algorithm}

\paragraph{Laufzeit} $O(n+m)$ durch perfekte Wahl mit topologischem Sortieren





%=====================================================
%=====================================================
%=====================================================
\section{SSSP-Algorithmus mit negativen Zyklen}
\label{sec-2}
Geben Sie einen SSSP-Algorithmus an, der negative Zyklen erkennt und ihn ausgibt.
Wie ist die Laufzeit?
\par
Bellman/Ford








%=====================================================
%=====================================================
%=====================================================
\section{Maxflow Labeling}
\label{sec-3}
Geben Sie einen Labeling-Algorithmus zur Lösung des Maxflow-Problems in Pseudo-Code an und analysieren Sie die Laufzeit.
\par












%=====================================================
%=====================================================
%=====================================================
\section{Capacity-Scaling}
\label{sec-4}
Geben Sie CAPACITY-SCALING unter Verwendungen des Maxflow Labeling Algorithmus an. Begründen Sie die Laufzeit.









%=====================================================
%=====================================================
%=====================================================
\section{Feasible Flow}
\label{sec-5}

\subsection{Berechnung des Feasible Flows}
Definieren Sie feasible flow. Formulieren Sie einen Algorithmus zur Berechnung eines feasible flows.


\subsection{Verwendung bei Maxflow}
 Wie kann man Maxflow damit lösen?




%=====================================================
%=====================================================
%=====================================================
\section{Min-Cost-Flow-Problem}
Wie lautet die reduzierte Kosten-Optimalität für das Min-Cost-Flow-Problem? Folgern Sie aus ihrer Gültigkeit, dass im Restnetzwerk keine negativen Zyklen existieren.

















