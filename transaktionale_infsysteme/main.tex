\documentclass[ngerman]{scrartcl}
\usepackage{amsmath,amsthm,amssymb}
\usepackage[T1]{fontenc}
\usepackage[utf8]{inputenc}
\usepackage{lmodern}

\usepackage{hyperref}

\title{Transaktionale Informationssysteme \\ SoSe19}
\author{Benedikt Lüken-Winkels}
\begin{document}

\maketitle
\tableofcontents
\newpage
\begin{abstract}
\end{abstract}

\section{1. Vorlesung}
Foliensatz 1
\subsection*{Orga}
\begin{itemize}
  \item \textbf{Vorlesung} Di, 14:15-15:45, H11
  \item \textbf{Übung} Mo, 13-14
  \item \textbf{Prüfung} mündlich 16.06. und 22.10.
\end{itemize}
\subsection*{Motivation}
Bei vielen, kurzen Transaktionen (Änderungen) darf die Datenbasis nicht zerstört werden
\begin{itemize}
  \item Rollback
  \item Administration der Aktionen auf der Datenbasis
  \item $\Rightarrow$ Datenkonsistenz
\end{itemize}
\paragraph*{Konsistenz}
Bewahrung der Korrektheit Daten im Fehlerfall

\paragraph*{Generizität}
Abstraktion von Szenarien

\subsection*{Paralleler Zugriff Beispiel 1.1 (Folie 12)}
Naive Parallelverarbeitung sorgt zum Konflikt

\paragraph*{Optimistischer Ansatz}
Laufen lassen, bis ein Fehler Auftritt

\paragraph*{Pessimistische Ansatz}
Zugriff blockieren

\subsection*{Fehlerhafte Ausführung Beispiel 1.2 (Folie 13)}
Prozess wird durch Fehler unterbrochen

\paragraph*{Rollback}
Sollten nicht alle Aktionen ausführbar sein, nicht ausführen (Komplett oder gar nicht)

\subsection*{Verteiltes Datensystem Beispiel 1.3 (Folie 14)}
Verschiedene Datenbestände nicht korrekt synchronisiert (zB Client- und Serverwarenkorb), Datensysteme sind verschieden und unahängig voneinander (heterogen und autonom)

\paragraph*{Transaktionale Eigenschaften}
\begin{itemize}
  \item Synchronisierung von Client und Serverinformationen
  \item Verifikation des Abschlusses einer Transaktion
\end{itemize}

\subsection*{Beispiel 1.4 (Folie 19)}
Gesamte Aktion muss erfolgreich sein: Schlägt eine Transaktion im Block fehl, wirf eine Fehlermeldung (zB Prüfungsanmeldung und Bestätigung)

\subsection*{Workflow Management}
Spezifikation von Workflows 
\begin{itemize}
  \item Wer bekommt welche Rolle
\end{itemize}

\paragraph*{Workflow}
\begin{itemize}
  \item Geschäftsprozess (zB Beschaffung, Reiseplanung) 
  \item Langlebig
\end{itemize}
\paragraph*{Aktivität}
Teile eines Workflows, die von verschiedenen Akteuren augeführt werden

\subsection*{Architekturen}
\paragraph*{Einfache Server Struktur (Folie 27)}
Data Server: Datendatendarstellung
\begin{itemize}
  \item Gekapselt in Objekten (Request, Reply)
  \item Ungekapselt als Tupel 
\end{itemize}
\paragraph*{Föderierte Systeme}
\begin{itemize}
  \item Alte Systeme müssen mit neuen Systemen kooperieren
\end{itemize}

\subsection*{Transaktionsmanagement}
\paragraph*{ACID (Folie 30)}
\begin{itemize}
  \item Atomarität: Ganz oder gar nicht
  \item Consistenz: Konsistenzerhaltung, waren die Daten Konsistent vor der Transaktion, sind sie es auch danach
  \item Isolation: Transaktionen beeinflussen sich nicht gegenseitig
  \item Dauerhaftigkeit: Wenn Transaktion erfolgreich, so ist sie in der Datenbank vorhanden 
\end{itemize}
\paragraph*{Anforderungen and Transaktionsmanagement (Folie 31)}
\begin{itemize}
  \item Concurrency Control
  \item !nachgucken!
\end{itemize}

\paragraph*{Aufbau (Folie 32)}
\begin{itemize}
  \item Transaktionsmanagement sorgt für Synch der Zugriffe
  \item Datenbank-Cache: Lesen und Bearbeiten der Daten im DB-Cache. Schreiben geschieht später
  \item DB Seiten (Folie 37)
\end{itemize}


\section{2.Vorlesung}
Transaktion ist eine Sequenz von Operationen

\subsection*{Partiell geordnete Transaktionen}
\begin{itemize}
  \item Reflexiv
  \item Antisymmetrisch
  \item Transitiv
\end{itemize}
\subsection{Objekt-Modell}
Baumdarstellung einer Transaktion (welche Methode ruft welche Methode auf)
\begin{itemize}
  \item Baum mit endlicher Höhe
  \item Innere Knoten sind Namen und Parameter von Operationen
  \item Blätter sind Seitenoperationen
  \item Besteht eine Ordung auf einer Ebene, so ist die höhere Ebene auch geordnet
\end{itemize}

\subsection{Concurrency Control}
\subsubsection*{Klassische Synchprobleme}
Verlust von ACID Eigenschaften, wenn Transaktionen unkontrolliert ausgeführt werden.
\begin{itemize}
  \item Lost Update Problem: Keine Kommunikation zwischen Prozessen während eines Updates
  \item Inconsistent Read Problem: Transaktion$ _1 $ liest während Transaktion$ _2 $ läuft $ \Rightarrow $ Änderungen nicht abgeschlossen 
  \item Dirty Read Problem: Ein Zwischenwert einer Transaktion wird für andere Transaktionen lesbar $ \Rightarrow $ Abbruch der Transaktion sorgt für Fehler durch gelesenen Wert 
\end{itemize}

\subsubsection*{Schedules}
Modell für verschränke/parallele Ausführung von Transaktionen
\begin{itemize}
  \item Transaction Manager entscheidet endgültig über die Reihenfolge in der Ausführung der Transaktionen verschiedener Prozesse
  \item Abort einer Transaktion: Rückführung aller Effekte
  \item Commit einer Transaktion
  \begin{itemize}
    \item Abgeschlossen und Effekte für andere Transaktionen sichtbar 
    \item Muss atomar durchlaufen (ganz oder gar nicht)
  \end{itemize}
\end{itemize} 
\paragraph*{Historie}
\begin{itemize}
  \item Vollständige Darstellung aller Transaktionen (mit Information über Commit und Abort) 
  \item In der Praxis ist nur der erste Ausschnitt (Präfix) sichtbar $ \Rightarrow $ Schedule
  \item Serielle Historie: Transaktionen werden hintereinander ausgeführt (nicht parallel)
  \item Nicht-serielle Historie: Verschränkte Ausführung von Transaktionen 
\end{itemize}
\paragraph*{Shuffle Produkt} Mischen von Transaktionen, die verschiedene Ausführungsreihenfolgen ergeben. Shuffle Produkt ist gültig, wenn 
\begin{itemize}
  \item Reihenfolge der Operationen beibehalten
  \item Keine Operationen hinzufügen
  \item Keine Operation entfernen
\end{itemize} 
Aus der Menge der gültigen Produkte muss der Scheduler ein Element (eine Ausführungsmöglichkeit) wählen.

\paragraph*{Total geordneter Schedule}
Entscheidungen werden auf Grund des Wissens aus den Präfixen einer Historie getroffen.

\paragraph*{Partiell geordnete Historien}
Für Histoire S gilt:
\begin{itemize}
  \item S ist vollständig: Transaktionen und Ergebnis (commit/abort)
  \item Eine Transaktion in S ist entweder commited oder aborted
  \item Ist eine Transaktion geordnet, so ist sie es auch in S
  \item Ordnung verschiedener Transaktionen auf dem gleichen Objekt
\end{itemize}
S 



\end{document}

