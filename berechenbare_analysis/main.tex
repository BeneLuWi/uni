\documentclass[ngerman]{scrartcl}
\usepackage{amsmath,amsthm,amssymb}
\usepackage[T1]{fontenc}
\usepackage[utf8]{inputenc}
\usepackage{lmodern}
\usepackage{graphicx}

\usepackage{hyperref}

\title{Berechenbare Analysis \\ SoSe 19}
\author{Benedikt Lüken-Winkels}
\begin{document}

\maketitle
\tableofcontents
\newpage


\section{Vorlesung}
\section{Vorlesung}
\subsection{Berechenbarkeit}
Es gibt einen Algorithmus, der die Zahl angeben kann (es gibt nur eine abzählbar unendliche Anzahl an Algorithmen, aber überabzählbar viele reelle Zahlen)
\begin{figure}[h!]
  \centering
  \caption{ g ist $ (\nu_x, \nu_y) berechenbar $, wenn g von einer berchenbaren Funktion f realisiert wird}
  \includegraphics[width=0.5\textwidth]{berechenbarkeit}
\end{figure}
\subsection{Entscheidbarkeit}
\paragraph{Diagonalisierung} Wären die Reellen Zahlen abzählbar, wäre die Diagonalzahl darin enthalten (!Widerspruch).
\begin{table}[h!]
  \caption{Diagonialisierungsbeispiel: $ x_\infty $ kann nicht in der Liste enthalten sein}
  \label{tab:diagonal}
  \begin{center}
    \begin{tabular}{cc}
       $ x_0 $ & 0.500000 \\
       $ x_1 $ & 0.411110 \\
       $ x_2 $ & 0.312110 \\
       $ x_3 $ & 0.222220 \\
       $ x_4 $ & 0.233330 \\
       ... & ... \\
       \hline\\
       $ x_\infty $ & 0.067785....
    \end{tabular}
  \end{center}
\end{table}

\paragraph{Definition}
 Menge A Entscheidbar, wenn eine Funktion $ f_{A}(x) $ ,die entscheidet, ob $ x \in A $ berechenbar ist.

\subsection{Berechenbare Reelle Zahlen}

\paragraph{Konstruktive Mathematik}
Formulierung algorithmischen Rechnens: zB $ \exists $ neu definiert als "es existiert ein Algorithmus". Nicht mehr für "klassische Mathematiker" lesbar

\paragraph{Definition}
Für $ x \in \mathbb{R} $ sind die Bedingungen äquivalent (wenn eine Bedingung erfüllt ist, sind alle Erfüllt):
\begin{enumerate}
  \item Eine TM erzeugt eine unendlich lange binäre Representation von $ x $ auf dem Ausgabeband
  \item \textbf{Fehlerabschätzung} Es gibt eine TM, die Approximationen liefert. Formal: $ q:\mathbb{N}\rightarrow \mathbb{Q} $ $ (q_{i})_{i \in \mathbb{N}} $ ist Folge rationaler Zahlen, die gegen $ x $ konvergiert. Bedeutet, dass alle $ q_i $ innerhalb eines bestimmten beliebig kleinen Bereichs um $ x $ liegen. Größter möglicher Fehler $ 2^0 = 1 $
  \item \textbf{Intervalschachtelung} Es gibt eine berechenbare Intervallschachtelung: Angabe zweier Folgen rationaler Zahlen mit der Aussage, dass $ x $ dazwischen liegt. Ziel: Abstände von linker und rechter Schranke soll gegen null gehen.
  \item \textbf{Dedekindscher Schnitt}Menge $ \{q \in \mathbb{Q} | q < x \} $ ist entscheidbar. Beispiel $ \sqrt{2} $ ist berechenbar. $ \{ q | q < \sqrt{2} \} = \{ q | q^2 < 2\}$. $ \Rightarrow $ Es gibt einen Test, ob die Zahl kleiner ist.
  \item $ z \in \mathbb{Z} $ $ A \subseteq \mathbb{N} $, $ x_A = \sum{i \in A} 2^-1-i $, $ x = z + x_A $
  \item Es exisitert eine Kettenbruchentwicklung
 \end{enumerate}
 \paragraph{Folgerungen / Beispiele}
\begin{itemize}
  \item $ \Rightarrow $ Für Berechenbarkeit muss nur eine der Bedingungen bewiesen werden. Menge der berechenbaren reelen Zahlen = $ \mathbb{R}_c $
  \item Nicht berechenbare reele Zahlen durch Diagonalisierung konstruierbar
  \item $ e $ berechenbar, weil die Fehlerabschätzung (2) existiert
  \item $ \pi $ (Notiert als alternierede Reihe) berechenbar, weil Intervalschachtelung existiert
  \item $ \sqrt{2} $ berechenbar, weil Dedekindscher Schnitt existiert.
 \end{itemize}

\paragraph{Implementierung}
Ziel: zB Berechnung von Differentialgleichungen 

\section{Vorlesung}
\paragraph{Implementierung in C++}
Ziel: shared pointer für temporäre Variablen verstecken (durch wrapper)
\begin{itemize}
  \item (binary sequence) bs: ein Bit nach dem anderen wird ausgegeben. binseq gibt zur natürlichen Zahl n und liefert das n-te Bit der reellen Zahl (Vorzeichen, 0 oder 1).
  \item (rational approximations) ra: Fehler beliebiger Größe (Gnaze Zahlen). approx rationale Approximation mit einem beliebig großem Fehler. (Abänderung der Definition, weil ganze Zahlen zulässig)
  \item ni: Untere und obere Schranke. lower/upperbound gibt n-te Schranke
  \item (Dedekind cut) dc: Ist eine Zahl kleiner. smaller entscheidet, ob die angegebene Rationale Zahl kleiner ist.
  \item ds: decide ist das n-te Bit gesetzt oder nicht
  \item cf: cont-fraction n-tes Folgenglied
\end{itemize}

\subsection{Binary Sequence}
\begin{itemize}
  \item make-node erzeugt den shared pointer auf das node Objekt
  \item DAG (directed acyclic graph) als Stuktur für Operatoren
\end{itemize}

\section{Vorlesung}
Programmierung


\section{Vorlesung}
\subsection{(2) $ \Rightarrow $ (1)}
Umsetung von Approximation zur Binärfolge für die gesuchte Zahl x:
\begin{itemize}
  \item Bereich zwischen 2 ganzen Zahlen aproximieren (ist x eine 2er-Potenz, schlägt dieser Schritt fehl). Fallunterscheidung:
  \begin{itemize}
    \item Ist die Zahl ein endlicher Binärbruch schreibe diesen auf
    \item ,sonst appoximiere und schreibe dann den endlichen Binärbruch
  \end{itemize}
  \item Binärsequenzen eignen sich nicht zum Rechnen
\end{itemize}

\subsection{$ \mathbb{R}_c $ ist ein Körper}
\begin{itemize}
  \item Sind 2 Zahlen berechenbar, so auch das Ergebnis aus + - * / $ \Rightarrow $ gilt für Intervallschachtelungen (Lemma 3.8)
  \begin{itemize}
    \item + : untere/obere Grenze addieren
    \item - : untere/obere Grenze subtrahieren
    \item * , / : min und max des Kreuzproduktes
  \end{itemize}
  \item Ein Polynom mit berechenbaren Koeffizienten hat berechenbare Nullstellen
\end{itemize}

\section{Vorlesung}

\subsection{DAG}
Interne Datenstruktur der Zahlen
\begin{itemize}
  \item Auswertung der Zahlenwerte nur bei Bedarf (lazy eval)
  \item Bei einer Berechnung wird ein neuer "Rechenknoten" mit Pointer auf die Variable erstellt
  \begin{itemize}
    \item Ein Knoten pro Operation (sehr Speicherintensiv)
    \item Lösung: Komplexere Rechenknoten
  \end{itemize}
\end{itemize}

\subsection{Berechenbare reelle Folgen}
\paragraph{Berechenbarkeit einer Folge} 
\begin{itemize}
  \item Berechenbare Folge berechenbarer Zahlen
  \item Das n-te Folgenglied der Folge kann mit Fehler $ 2^-i $ durch eine berechenbare Folge rationaler Zahlen approximiert werden
  \item Nicht alle reellen Zahlen können durch eine berechenbare reelle Folge berechnet werden
  \begin{itemize}
    \item Wähle eine rationale Folge $ q_n $, die $ x_n $ approximiert
    \item Wähle $ x_n $ so, dass es außerhalb dem approximierten Bereich von $ q_n $ liegt (Diagonalisierung)
  \end{itemize}
\end{itemize}

\section{Vorlesung}
NACHTRAGEN


\section{Vorlesung}
\subsection{Struktur berechenbarer Funktionen}
\subsubsection{Orakel-Turingmaschine OTM}
\begin{itemize}
  \item Turingmaschine mit Zugriff auf eine Orakelfunktion $ \phi $
  \item Ein Zustand ist Orakelzustand $ s_O $
  \item Ein Band ist Orakelband
  \item Geht die Maschine in den Zustand $ s_O \Rightarrow$ (partielle) Orakelfunktion wird aufgerufen:
  \begin{itemize}
    \item Eingabe auf Orakelband wird evaluiert = 'Anfrage an das Orakel'
    \item $ v \in Def(\phi) $: Orakelfunktion schreibt Antwort auf Orakelband in einem Schritt
    \item $ v \not\in Def(\phi) $: Orakelfunktion endet mit Fehler
  \end{itemize}
  \item Orakel kann zB benutzt werden, um das Halteproblem entscheiden. Das richtige Orakel, kann P=NP simulieren.
  \item $ f^\phi _M $ Berechnete Funktion 
  \item $ T^\phi _M(w) $ Anzahl der Rechenschritte 
  \item $ A^\phi _M(w) $ Menge der Angfragen    
\end{itemize} 
Menge der von OTM berechenbaren Funktionen ist $ \mathbb{F} $. Typ-2-Mengen zB $ \mathbb{R}\ \mathbb{N}^\mathbb{N} \Rightarrow$ Überabzähbar. Typ-1-Mengen $ \Rightarrow $ abzählbar unendlich

\subsubsection{Typ-2-Turingmaschinen}
\begin{itemize}
  \item spezielles Ein/Ausgabeband (Eingabe: read-only, Ausgabe: one-way = Ausgabe nicht mehr modifizierbar)
  \item Arbeitsweise wie eine normale TM
  \item Eingabe darf unendlich lang sein
  \item Ausgabe endlich, wenn die Maschine hält oder läuft unendlich 
  \item $ T_M(p)(n) $ Anzahl der Rechenschritte bis Ausgabe des Zeichens $ q_n $ 
  \item $ A_M(p)(n) $ Anfragenlänge zur Berechnung bis zum Zeichen von $ q_n $    
\end{itemize}

\subsubsection{Zusammenhände OTM und Typ-2-TM}
Unendliche Eingabe aus Typ-2-TM wird durch Orakel zu einer Näherung, um von OTM verarbeitet werden zu können. So kann eine OTM eine Typ-2-TM simulieren. 


\section{Vorlesung}
NACHTRAGEN vom 24.05.
\begin{itemize}
  \item Darstellung für unendl Folgen von Zeichen oder Wortfunktionen von Strings auf Strings
\end{itemize}


\section{Vorlesung}
\subsection{Cauchy-Darstellung}
$ M = [ \mathbb{N} \rightarrow \mathbb{Q} ] $
\begin{itemize}
  \item Implementierung Folgen rationaler Zahlen mit gewissen Näherungen
  \item Cauchy-Folge: der Abstand zweier Folgeglieder ist kleiner, als ein Schwellenwert
  \item $ \rho $ sind die schnell konv rationalen Folgen
  \item Enthält unberechenbare Folgen
  \item Erfasst alle berechenbaren reelen Zahlen über berechenbare Namen 
\end{itemize}
\paragraph{Beispiel Notation von $ f(x) = 3x $}
Die Typ-2-TM M kann einen der Namen für die Eingabe ausgeben. Namen für 1: 0.9999... und 1.0000... . Fallunterscheidung:
\begin{enumerate}
  \item Ab einem bestimmten Punkt ist p'=w999... und ergibt 1.00..2000
  \item Ab einem bestimmten Punkt ist p'=w000... und ergibt 0.99..9000
\end{enumerate}
$ \Rightarrow $ Nicht berechenbar, wenn $ \delta_{dez} \rightarrow \delta_{dez} $ Abgebildet wird. Berechenbarkeit kann nur durch andere Abbildungsmenge erreicht werden, wie Cauchy ($ [\mathbb{N}\rightarrow\mathbb{Q}]$)

\section{Vorlesung}
NACHTRAGEN
linksberechenbare/rechtsberechenbare Zahlen

\section{Vorlesung}
Stetig berechenbare Funktionen
\subsection{Metrischer Raum}
\begin{itemize}
  \item d(x,y) Abstand zweier Punkte. Nahegelegene Punkte finden. Hilfreich für Cauchy-Darstellung um andere genäherte Zahlen zu finden, die sich auch innerhalb des Fehlers liegt.
  \item $ B(x,\epsilon) $ Formale Kugel: Mittelpunkt, Radius: Gibt alle Punkte mit Abstand kleiner, als der Radius. 
  \item $ B^n $ alle Formale Kugeln, wo Zentrum und Radius $ \in\mathbb{Q} $. Zahl in 3 Komponenten als Kantorsche Zerlegung: (Zentrum, Radius)
\end{itemize}

\paragraph{Effektiv stetig} 
$ S \subseteq \mathbb{N} $ ist rekursiv aufzählbar. Eine Funktion ist genau dann berechenbar, wenn sie effektiv stetig ist. 07.06. NACHHÖREN für den Beweis
\begin{enumerate}
  \item $ <i,j> \in S\ mit\ f(B^n(x))\subseteq B^1(j) $ erzeugt Rechtecke, durch die die Funktion laufen muss. Die Funktion liegt innerhalb der Schläuche.
  \item für jedes $ x\in Def(f) $ kann man ein $ <i,j> \in S $ finden. Die Schläuche werden beliebig fein.
\end{enumerate}

\paragraph{Folgerungen} Vorzeichenfunktion ist nicht stetig. $ sign: \mathbb{R} \rightarrow \mathbb{R}  $ ist nicht implementierbar (nicht berechenbar). Berechenbarkeit wird durch sign' erreicht, indem die Funktion partiell wird, indem sign' bei $ x=0 $ in eine Endlosschleife läuft. $ \overline{sign} $ ist total und bb, wenn, wenn x um 0 liegt $ \overline{sign} $ 0 oder 1 ausgibt.  

\section{Vorlesung}
\subsection{Mehrwertige Funktionen}
Mehrwertige Funktion $ f: \subseteq X \rightrightarrows  Y $ ist eine Funktion, die für ein x mehrere Werte für y haben kann. Ein Funktionswert eines x sind alle möglichen Werte aus Y.

\paragraph{Komposition von mehrwertigen Funktionen} In allen Fällen, muss das Ergebnis definiert sein. Definitionsbereich  der Komposition $ f \cdot g $ sind die x und y, die in beiden Funktionen im Definitionsbereich liegen. Außerhalb des Definitionsbereichs dürfen die Funktionen 'machen was sie wollen'. Beispiele:
\begin{itemize}
  \item In der Implementierung: Approx von 2 und $ \sqrt{2} * \sqrt{2} $.
  \item Konversion von $ \mathbb{R} $ in Dezimalzahlen. Rundung mit erlaubter Schwankung ergibt verschiedene Ausgaben. Eindeutige Umwandlung (Rundung) ist nicht berechenbar, aber mehrwertig bb.
\end{itemize}

\paragraph{Konstruierte Folgen $ (x_n)_n $} nicht-bb Grenzwert, aber monoton wachsend. Nicht berechenbarer Konvergenzmodul. \\
Die Funktion f ist auf den bb reellen Zahlen stetig, aber nicht bb mit einer nicht-bb kleinsten Nullstelle. Definitionsbereich von f ist $ \mathbb{R} ohne \{x_A \} $, also nicht stetig auf $ x_A $. Eigenschaften von f sind abhängig von A :
\begin{itemize}
  \item Ist A entscheidbar und der Definitionsraum ohne $ x_A $ ist berechenbar. (Sonst ist A ist so kompliziert, wie das Halteproblem und $ x_A $ kodiert das Halteproblem in einer reellen Zahl)
  \item A ist rekursiv-aufzählbar, aber nicht entscheidbar. f bildet die bb reellen Zahlen auf die bb reellen Zahlen ab. 
\end{itemize}
 


%=======================================
%
%
% Übungen
%
%
%=======================================

\section{1. Übung}
\subsection*{Aufgabe 1}
\begin{itemize}
  \item Ziel: Finden des richtigen n für den Fehler
  \item Die Größe des Unterschieds zwischen x und y muss größer sein, als die Summe der Fehler
  \item Gleichheit testen geht nicht mit einer totalen Funktion
\end{itemize}
\subsection*{Aufgabe 2}
$ (4) \Rightarrow (3) $
\begin{itemize}
  \item Menge der kleineren Zahlen ist entscheidbar
  \item Durchtesten der Integers ob die Zahlen innerhalb oder außerhalb der Menge liegen  
  \item Aus der Entscheidbarkeit der Menge werden die Folgen für die Schranken
\end{itemize}
$ (3) \Rightarrow (2) $
\begin{itemize}
  \item Differenz zwischen den Schranken ergibt Fehlergröße
  \item Folge q ist die die Folge, die sich aus der Mitte $ \frac{a + b}{2} $
\end{itemize}
$ (2) \Rightarrow (3) $
\begin{itemize}
  \item Schranken a, b ergeben sich aus Folge +/- Fehler
  \item $ a_k = max(q_k + 2^-k - \frac{1}{k}, a_k-1) $
  \item $ b_k = min(q_k - 2^-k - \frac{1}{k}, a_k-1) $
\end{itemize}
$ (2/3) \Rightarrow (4) $ \\
Zusätzlicher Test, wenn die Zahl rational ist, weil der Test auf Gleichheit eine Endlosschleife

\subsection*{Aufgabe 3}
$ x_A bb \Rightarrow A_{entscheidbar} $\\
Tablemakers dilemma

\section{3. Übung}
\subsection*{1. Aufgabe}
\paragraph{Identität auf den reellen Zahlen ist nicht $ (\rho, \delta_{dez}') $-berechenbar} 
\paragraph{Identität auf den reellen Zahlen ist $ (\delta_{dez}', \rho) $-berechenbar} Nimm eine Kommastelle nach der Anderen und Formuliere die Rationale Zahl

\subsection*{2.Aufgabe} 
\paragraph{max} Problem bei Gleichheit


\section{4. Übung}




%=======================================
%
%
% Definitionen
%
%
%=======================================
\section{General Stuff}
\begin{itemize}
  \item Abzählbar unendlich, wenn Bijektion zu $ \mathbb{N} $ $ \Rightarrow $ So viele berechenbare reelle Zahlen, wie Programme
  \item Entscheidbar: Eine Funktion gibt aus, ob ein Element aus einer Menge ist.
  \item Rekursiv Aufzählbar: Es gibt eine TM, die anhält, wenn ein Element aus der Menge ist.
  \item Berechenbarkeit $ (\rho, \rho)-bb $ '(Approximation, Approximation)-bb'
  \begin{itemize}
    \item Eine Funktion $ f: \subseteq \mathbb{R}^n \rightarrow \mathbb{R} $ ist genau dann berechenbar, wenn sie effektiv stetig ist.
    \item Aus Stetigkeit, Berechenbarkeit. Aus nicht Stetigkeit folgt nicht Berechenbarkeit
    \item Um nicht-Berechenbarkeit zu zeigem, zeigt man nicht-Stetigkeit 
    \item Aus nicht-effektiver Stetigkeit folgt nicht-Berechenbarkeit. Bsp $  f(x) = 1, x \geq 0; 0, x < 0 $ ist nicht stetig und nicht berechenbar
  \end{itemize}
  \item Konvergenzmodul
  \item Effektiv stetig
  \item Diagonialisierung
  \item Isomorphismus
  \item Homomorphismus
  \item Berechenbarkeits-Typen
  \item Cantorsche Zerlegung/Cantorsche Paarungsfunktion
  \item Aus effektiver Stetigkeit folgt Stetigkeit. 
  \item Cauchy-Darstellung
  \item Warum kann nicht auf 0 geprüft werden.
\end{itemize}


\end{document}
