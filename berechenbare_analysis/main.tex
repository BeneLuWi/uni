\documentclass[ngerman]{scrartcl}
\usepackage{amsmath,amsthm,amssymb}
\usepackage[T1]{fontenc}
\usepackage[utf8]{inputenc}
\usepackage{lmodern}
\usepackage{graphicx}

\usepackage{hyperref}

\title{Berechenbare Analysis \\ SoSe 19}
\author{Benedikt Lüken-Winkels}
\begin{document}

\maketitle
\tableofcontents
\newpage


\section{Vorlesung}
\section{Vorlesung}
\subsection{Berechenbarkeit}
Es gibt einen Algorithmus, der die Zahl angeben kann (es gibt nur eine abzählbar unendliche Anzahl an Algorithmen, aber überabzählbar viele reelle Zahlen)
\begin{figure}[h!]
  \centering
  \caption{ g ist $ (\nu_x, \nu_y) berechenbar $, wenn g von einer berchenbaren Funktion f realisiert wird}
  \includegraphics[width=0.5\textwidth]{berechenbarkeit}
\end{figure}
\subsection{Entscheidbarkeit}
\paragraph{Diagonalisierung} Wären die Reellen Zahlen abzählbar, wäre die Diagonalzahl darin enthalten (!Widerspruch).
\begin{table}[h!]
  \caption{Diagonialisierungsbeispiel: $ x_\infty $ kann nicht in der Liste enthalten sein}
  \label{tab:diagonal}
  \begin{center}
    \begin{tabular}{cc}
       $ x_0 $ & 0.500000 \\
       $ x_1 $ & 0.411110 \\
       $ x_2 $ & 0.312110 \\
       $ x_3 $ & 0.222220 \\
       $ x_4 $ & 0.233330 \\
       ... & ... \\
       \hline\\
       $ x_\infty $ & 0.067785....
    \end{tabular}
  \end{center}
\end{table}

\paragraph{Definition}
 Menge A Entscheidbar, wenn eine Funktion $ f_{A}(x) $ ,die entscheidet, ob $ x \in A $ berechenbar ist.

\subsection{Berechenbare Reelle Zahlen}

\paragraph{Konstruktive Mathematik}
Formulierung algorithmischen Rechnens: zB $ \exists $ neu definiert als "es existiert ein Algorithmus". Nicht mehr für "klassische Mathematiker" lesbar

\paragraph{Definition}
Für $ x \in \mathbb{R} $ sind die Bedingungen äquivalent (wenn eine Bedingung erfüllt ist, sind alle Erfüllt):
\begin{enumerate}
  \item Eine TM erzeugt eine unendlich lange binäre Representation von $ x $ auf dem Ausgabeband
  \item \textbf{Fehlerabschätzung} Es gibt eine TM, die Approximationen liefert. Formal: $ q:\mathbb{N}\rightarrow \mathbb{Q} $ $ (q_{i})_{i \in \mathbb{N}} $ ist Folge rationaler Zahlen, die gegen $ x $ konvergiert. Bedeutet, dass alle $ q_i $ innerhalb eines bestimmten beliebig kleinen Bereichs um $ x $ liegen. Größter möglicher Fehler $ 2^0 = 1 $
  \item \textbf{Intervalschachtelung} Es gibt eine berechenbare Intervallschachtelung: Angabe zweier Folgen rationaler Zahlen mit der Aussage, dass $ x $ dazwischen liegt. Ziel: Abstände von linker und rechter Schranke soll gegen null gehen.
  \item \textbf{Dedekindscher Schnitt}Menge $ \{q \in \mathbb{Q} | q < x \} $ ist entscheidbar. Beispiel $ \sqrt{2} $ ist berechenbar. $ \{ q | q < \sqrt{2} \} = \{ q | q^2 < 2\}$. $ \Rightarrow $ Es gibt einen Test, ob die Zahl kleiner ist.
  \item $ z \in \mathbb{Z} $ $ A \subseteq \mathbb{N} $, $ x_A = \sum{i \in A} 2^-1-i $, $ x = z + x_A $
  \item Es exisitert eine Kettenbruchentwicklung
 \end{enumerate}
 \paragraph{Folgerungen / Beispiele}
\begin{itemize}
  \item $ \Rightarrow $ Für Berechenbarkeit muss nur eine der Bedingungen bewiesen werden. Menge der berechenbaren reelen Zahlen = $ \mathbb{R}_c $
  \item Nicht berechenbare reele Zahlen durch Diagonalisierung konstruierbar
  \item $ e $ berechenbar, weil die Fehlerabschätzung (2) existiert
  \item $ \pi $ (Notiert als alternierede Reihe) berechenbar, weil Intervalschachtelung existiert
  \item $ \sqrt{2} $ berechenbar, weil Dedekindscher Schnitt existiert.
 \end{itemize}

\paragraph{Implementierung}
Ziel: zB Berechnung von Differentialgleichungen 

\section{Vorlesung}
\paragraph{Implementierung in C++}
Ziel: shared pointer für temporäre Variablen verstecken (durch wrapper)
\begin{itemize}
  \item (binary sequence) bs: ein Bit nach dem anderen wird ausgegeben. binseq gibt zur natürlichen Zahl n und liefert das n-te Bit der reellen Zahl (Vorzeichen, 0 oder 1).
  \item (rational approximations) ra: Fehler beliebiger Größe (Gnaze Zahlen). approx rationale Approximation mit einem beliebig großem Fehler. (Abänderung der Definition, weil ganze Zahlen zulässig)
  \item ni: Untere und obere Schranke. lower/upperbound gibt n-te Schranke
  \item (Dedekind cut) dc: Ist eine Zahl kleiner. smaller entscheidet, ob die angegebene Rationale Zahl kleiner ist.
  \item ds: decide ist das n-te Bit gesetzt oder nicht
  \item cf: cont-fraction n-tes Folgenglied
\end{itemize}

\subsection{Binary Sequence}
\begin{itemize}
  \item make-node erzeugt den shared pointer auf das node Objekt
  \item DAG (directed acyclic graph) als Stuktur für Operatoren
\end{itemize}

\section{Vorlesung}
Programmierung


\section{Vorlesung}
\subsection*{(2) $ \Rightarrow $ (1)}
Umsetung von Approximation zur Binärfolge für die gesuchte Zahl x:
\begin{itemize}
  \item Bereich zwischen 2 ganzen Zahlen aproximieren (ist x eine 2er-Potenz, schlägt dieser Schritt fehl)
  \item Binärsequenzen eignen sich nicht zum Rechnen
\end{itemize}

\subsection*{$ \mathbb{R}_c $ ist ein Körper}
\begin{itemize}
  \item Sind 2 Zahlen berechenbar, so auch das Ergebnis aus + - * / $ \Rightarrow $ gilt für Intervallschachtelungen (Lemma 3.8)
  \begin{itemize}
    \item + : untere/obere Grenze addieren
    \item - : untere/obere Grenze subtrahieren
    \item * , / : min und max des Kreuzproduktes
  \end{itemize}
  \item Ein Polynom mit berechenbaren Koeffizienten hat berechenbare Nullstellen
\end{itemize}

\section{Vorlesung}

\subsection*{DAG}
Interne Datenstruktur der Zahlen
\begin{itemize}
  \item Auswertung der Zahlenwerte nur bei Bedarf (lazy eval)
  \item Bei einer Berechnung wird ein neuer "Rechenknoten" mit Pointer auf die Variable erstellt
  \begin{itemize}
    \item Ein Knoten pro Operation (sehr Speicherintensiv)
    \item Lösung: Komplexere Rechenknoten
  \end{itemize}
\end{itemize}

\subsection*{Berechenbare reelle Folgen}
\paragraph{Berechenbarkeit einer Folge} 
\begin{itemize}
  \item Berechenbare Folge berechenbarer Zahlen
  \item Das n-te Folgenglied der Folge kann mit Fehler $ 2^-i $ durch eine berechenbare Folge rationaler Zahlen approximiert werden
  \item Nicht alle reellen Zahlen können durch eine berechenbare reelle Folge berechnet werden
  \begin{itemize}
    \item Wähle eine rationale Folge $ q_n $, die $ x_n $ approximiert
    \item Wähle $ x_n $ so, dass es außerhalb dem approximierten Bereich von $ q_n $ liegt (Diagonalisierung)
  \end{itemize}
\end{itemize}


\section{General Stuff}
\begin{itemize}
  \item Abzählbar unendlich, wenn Bijektion zu $ \mathbb{N} $ $ \Rightarrow $ So viele berechenbare reelle Zahlen, wie Programme
  \item Entscheidbar:
  \item Berechenbar: 
  \item Diagonialisierung
\end{itemize}


\end{document}







