\documentclass[ngerman]{scrartcl}
\usepackage{amsmath,amsthm,amssymb}
\usepackage[T1]{fontenc}
\usepackage[utf8]{inputenc}
\usepackage{lmodern}
\usepackage{multicol}

\usepackage{hyperref}

\title{Informationsvisualisierung \\ Zusammenfassung}
\author{Benedikt Lüken-Winkels}
\begin{document}

\maketitle
\tableofcontents
\newpage



\section{Einführung}
\paragraph{Definition} Informationsvisualisierung ist die Kommunikation von abstrakten Informationen durch interaktive visuelle Schnittstellen. 

\paragraph{Abgrenzung zur Visualisierung} Informationsvisualisierung ist nicht

\begin{itemize}
  \item Wissenschaftliche Visualisierung: Darstellung nicht-abstrakter Informationen mit realen physischen Representationen. (Röntgenbild)
  \item Computer Graphik: Technischer und Mathematischer Aspekt von Visualisierung
  \item Graphik-Design: Ästhetische graphische Darstellung
\end{itemize}


\paragraph{Darstellungsmöglichkeiten von abstrakten Daten oder Informationen}
\begin{itemize}
  \item Text und Tabellen
  \item Hierarchien und Graphen
  \item Multivariate Daten (Mehrdimensionale Daten)
  \item Zeitreihen (Multivariate Daten, wobei die Zeit eine besondere Dimension darstellt)
\end{itemize}


\section{Infographiken}

\subsection{Diagramme}

Einfache Beispiele von Diagrammen
\begin{itemize}
  \item Linien-Diagramm
  \item Balken-Diagramm
  \item Kuchen-Diagramm
  \begin{itemize}
    \item Gut bei Part-Whole-Relationen
    \item Tatsächliche Werte/Kategorien sind schwer zu vergleichen
  \end{itemize}
  \item Zeitreihen
  \item Sparkreihen: Zeitreihen, reduziert um Trends darzustellen
\end{itemize}


\subsection{Metaphern und Symbole}
\textbf{Metaphern} sollen Konstrukte und Konzepte verineinfacht darstellen und zugänglicher machen. Beispiele:
\begin{itemize}
  \item Städte (Cluster in Zusammenhängen)
  \item Bäume (Hierarchien)
  \item Tiere (Vererbeung)
\end{itemize}

\textbf{Symbole} stellen stark vereinfachte Darstellungen von Sachverhalten oder Objekten dar, die unter anderem auch Metaphern darstellen können (Papierkorb Icon für gelöschte Elemente).

\paragraph{Isotyp} Darstellung von statistischen Informationen durch Piktogramme (Symbole). Größe der Zahl wird durch Anzahl an Symbolen kodiert.

\subsection{Infographiken}
Infographiken sind graphische Representationen von Informationen, Daten oder Wissen, die komplexe Informationen schnell und leicht zugänglich machen sollen.

\subsubsection{Gegenüberstellung Infographik und Informationsvisualisierung}

\begin{multicols}{2}% 2-column layout
  \begin{minipage}{0.45\textwidth}
  \textbf{Infographik}
    \begin{itemize}
      \item Von Hand geschrieben
      \item Selbsterklärend
      \item Erzählt eine Geschichte
      \item Meistens Ad-Hoc
      \item Kann voreingenommen sein
    \end{itemize}
  \end{minipage}
  \textbf{Informationsvisualisierung}
    \begin{itemize}
      \item Automatisch generiert
      \item Auf verschiedene Datensätze anwendbar
      \item Nicht unbedingt selbsterklärend
    \end{itemize}
\end{multicols}

\section{Visuelle Wahrnehmung}
Visuelle Wahrnehmung ist selektiv, interpretierend und kann durch die Umgebung (Kontrast/Struktur) abgelenkt werden (nicht wie eine Kamera)

\subsection{Verarbeitung visueller Informationen}

\paragraph{Dreiphasenmodell} 
\begin{enumerate}
  \item Primitive Mustererkennung
  \item Aktionen/Reflexe, Komplexe Mustererkennung
  \item Visuelles Arbeitsgedächtnis
\end{enumerate}

\paragraph{Menschliches Auge} Man kann verschiedene Eigenschaften der visuellen Wahrnehmung ausnutzen. 
\begin{itemize}
  \item Antizipation von Bewegungen: Vorhersehen von Ereignissen
  \item Mustererkennung: Verdeutlichen von Clustern
\end{itemize}

\paragraph{Perphere Schärfe} Die Schärfe mit der Text lesbar oder Objekte erkennbar sind, nimmt abhängig vom Zentrum ab: 
\begin{itemize}
  \item Zentrum: Farbe und Helligkeit sind klar erkennbar.
  \item Rand/Peripherie: Unscharf und nur Helligkeit ist erkennbar.
\end{itemize}

\subsection{Farbwahrnehmung}

\paragraph{Opponent Color Theory. Gegenfarbtheorie}
Es ist einfacher und effizienter die Farben anhand der Unterschiede zwischen benachbarten Farben zu erkennen. 3 chemische Prozesse mit jeweils 2 Gegenfarben sorgen für Farbidentifizierung. $ \Rightarrow $ Farbe und Helligkeit sind relativ.



\subsection{Mustererkennung}
\paragraph{Kategorien von Mustern}
\begin{itemize}
  \item Kantenerkennung (Farben, Helligkeit)
  \item Einfache Muster (Verbindung durch Kanten oder Roation)
  \item Komplexe Muster (Muster in Mustern, Cluster)
  \item Objekterkennung (Icons, Symbole)
\end{itemize}


\paragraph{Präattentive Elemente} Ein Element in einer Gruppe ähnlicher Elemente mit herausstehenden Eigenschaften kann schnell erkannt werden. \textbf{Bewegung ist präattentiv}

\paragraph{Bewegungserkennung}
Bewegung kann auf einen Zusammenhang hinweisen. 


\paragraph{Visuelle Suche}
Folgt einem Zyklus:
\begin{enumerate}
  \item Erkenne Muster
  \item Wähle einen Kandidaten aus
  \item Schließe vorherige Ziele aus
  \item Bewege das Auge
\end{enumerate}

\paragraph{Gestalt Psychologie}
Gesetze, die helfen, Gruppierungen leichter erkennbar zu machen.
\begin{itemize}
  \item Räumliche Nähe
  \item Ähnlichkeit in Gestalt oder Form
  \item Verbindungen (durch Kanten in einem Graphen)
  \item Fortläufigkeit (Continuity, zB nicht unterbrochene Linien)
\end{itemize}



\subsection{Dreidimensionale Wahrnehmung}
Problematisch bei der Visualisierung von abstrakten Informationen. 
\paragraph{Hauptprobleme von 3D}
\begin{itemize}
  \item Schwierige Navigation
  \item Ausschluss von Information im Hintergrund
\end{itemize}

\paragraph{2,5D Darstellung}
Kombiniert die Vorteile von 2D und 3D. Dritte Dimension bietet Schatten oder Perspektive. Beispiele:
\begin{itemize}
  \item Cuschion treemap
  \item Cone Trees
  \item UML Geons
  \item Perspective Wall
\end{itemize}


\subsection{Design Empfehlungen}
\begin{itemize}
  \item Emphasize with color
  \item Differences with brightness
  \item Coding of categories: max 6 to 12 different colors
  \item Color scales should vary in color and brighntess 
  \item Color perception depends on culture
  \item Motion to grap attention/indicate a relation
  \item Strong colors/contrast can cause interta (ghost images)
  \item User yellow/blue variations for colorblinds
\end{itemize}


\section{Visualisierung von Hierarchien}

\paragraph{Hierarchische Daten/Informationen als Baumstruktur}
\begin{itemize}
  \item Keine Zyklen
  \item Es gibt genau einen Weg von der Wurzel zu einem beliebigen Knoten.
  \item $ \Rightarrow $ Verbundener, ungerichteter Graph 
\end{itemize}

\subsection{Node-Link}
Darstellung eines Baums, mit Knoten als Punkte und Kanten als Linien. \\
Vorteile: 
\begin{itemize}
  \item Intuitiv
  \item Hierarchie schnell erkennbar
  \item Flexibles Layout
\end{itemize}
Nachteile:
\begin{itemize}
  \item Kanten brauchen Platz
  \item Degernerierte Bäume sind schwer darzustellen.
\end{itemize}

\subsection{Eingerückte Gliederungs Plots}
Darstellungs eines Baums in einer Liste \\
Vorteile: 
\begin{itemize}
  \item Leicht lesbar
  \item Allgemein bekannt
  \item Darstellen degernerierter Bäume möglich
\end{itemize}
Nachteile:
\begin{itemize}
  \item Unflexibles Layout
\end{itemize}

\subsection{Eiszapfen Plot (Icicle Plot)}
Hierarchie als Teil-Ganzes Beziehung \\
Vorteile: 
\begin{itemize}
  \item Labelling ist einfach
  \item Effektive Nutzung von Bildschirmplatz
\end{itemize}
Nachteile:
\begin{itemize}
  \item Nicht sehr intuitiv
  \item Breite der Kindelement wird durch Breite der Vaterelemente beschränkt
\end{itemize}

\subsection{Treemap}
Elemenete als Flächen (Venn Diagramm - artig) \\
Vorteile: 
\begin{itemize}
  \item Kaum Platzverbrauch für Kanten
  \item Platz der Blattknoten kann verbraucht werden
\end{itemize}
Nachteile:
\begin{itemize}
  \item Labelling schwierig
  \item Rand eines Elements muss berücksichtigt werden und ins Layout mit einbezogen werden. (2,5D)
\end{itemize}



\subsection{Verwendung in der Praxis}
Empfohlene Layouts: Node-Link, Einzapfen, Eingerückte Gliederung. Nicht empfohlen, weil unintuitiv und meistens schwer lesbar: Treemap und Radial Layouts.
 
\paragraph{Vergleich von Hierarchien} Aufbau als Matrix, farbliche Hervorhebung, Linienverbindungen


\section{Visualisierung von Graphen}

\paragraph{Anwendung}
\begin{itemize}
  \item Karten: Multivariate Daten in einer Karte (London Underground: Verbindungen und geographische Lage)
  \item Ego/Personen-Netzwerke: Persönliche Verbindungen als Graph (Facebook)
\end{itemize}

\paragraph{Visuelle Kodierung}
\begin{itemize}
  \item Dicke einer Kante: Kantengewicht
  \item Pfeil: Kantenrichtung
  \item Verschachtelte Boxen: Hierarchie
  \item Knotenlabel: Knotennamen
\end{itemize}

\paragraph{Ästhetische Kriterien}
\begin{itemize}
  \item Minimale Kantenüberschneidung
  \item Minimale Bemalte Fläche
  \item Minimale Kantenlänge
  \item Minimale Anzahl an Biegungen in Kanten
  \item Maximale Symmetrie
  \item Verdeutlichung von Clustern
  \item Fortlaufende Winkel an den Kanten
\end{itemize}


\subsection{Layout Algorithmen}

\paragraph{Radial Layout}
\begin{itemize}
  \item Gleichverteilung im Knotengewicht
  \item Viele Kantenüberschneidungen
  \item Gut anwendbar, wenn keine Metadaten vorhanden sind
\end{itemize}

\paragraph{Force-Directed Layout}
\begin{itemize}
  \item Kanten werden wie Federn in eine bestimmte Länge geforcet
  \item Equilibrium, wo sich Kantenkräfte gegenseitung ausbalancieren
\end{itemize}

\paragraph{Hierarchisches Layout}
\begin{itemize}
  \item Problem beim Zeichnen von Zyklen:
  \begin{itemize}
    \item Drehe die Kanten um, die den Zyklus schließen
    \item Male dann den Graphen
    \item Drehe die Kanten wieder um
  \end{itemize}
  \item Tiefensuche ergibt die topologische Ordnung der Elemente
  \item Dummy Knoten, die nach dem Erstellvorgang entfernt werden ergeben ein sauberes Layout
\end{itemize}


\paragraph{Orthogonales Layout}
\begin{itemize}
  \item Kanten folgen einem Raster
  \item Metriken für die Form: zB Je weniger Wendungen in einem Pfad, desto besser
\end{itemize}

\paragraph{Kantenbündelung}
\begin{itemize}
  \item Strukturiertes Radial Layout
  \item Kanten, die in die selbe Richtung gehen werden gebündelt
\end{itemize}

\subsection{Matrix Visualisierung von Graphen}

\paragraph{Adjazenzmatrix} NxN-Matrix, mit Einfärbungen, wenn eine Kante existiert. Clustererkennung ist schwer. $ \Rightarrow $ Sortierung der Knoten


\subsection{Visualisierung von dynamischen Graphen}
Zeitliche Abfolge von Graphzuständen

\paragraph{Animation} see difference between layout and data changes to preserve the mental map of the graph. Examples:
\begin{itemize} 
  \item TimeArchTrees, horizontal development of the graph, vertical orientation of the graph  
  \item TimeSpiderTrees, cirular layout, each ring is one graph
  \item TimeRadarTrees, cicular layout, outer circles are a representation of the inner. The inner circle shows incoming edges, the outer shows outgoing
\end{itemize}

\end{document}















