\documentclass[ngerman]{scrartcl}
\usepackage{amsmath,amsthm,amssymb}
\usepackage[T1]{fontenc}
\usepackage[utf8]{inputenc}
\usepackage{lmodern}
\usepackage{multicol}

\usepackage{hyperref}

\title{Informationsvisualisierung \\ Zusammenfassung}
\author{Benedikt Lüken-Winkels}
\begin{document}

\maketitle
\tableofcontents
\newpage



\section{Einführung}
\paragraph{Definition} Informationsvisualisierung ist die Kommunikation von abstrakten Informationen durch interaktive visuelle Schnittstellen. 

\paragraph{Abgrenzung zur Visualisierung} Informationsvisualisierung ist nicht

\begin{itemize}
  \item Wissenschaftliche Visualisierung: Darstellung nicht-abstrakter Informationen mit realen physischen Representationen. (Röntgenbild)
  \item Computer Graphik: Technischer und Mathematischer Aspekt von Visualisierung
  \item Graphik-Design: Ästhetische graphische Darstellung
\end{itemize}


\paragraph{Darstellungsmöglichkeiten von abstrakten Daten oder Informationen}
\begin{itemize}
  \item Text und Tabellen
  \item Hierarchien und Graphen
  \item Multivariate Daten (Mehrdimensionale Daten)
  \item Zeitreihen (Multivariate Daten, wobei die Zeit eine besondere Dimension darstellt)
\end{itemize}


\section{Infographiken}

\subsection{Diagramme}

Einfache Beispiele von Diagrammen
\begin{itemize}
  \item Linien-Diagramm
  \item Balken-Diagramm
  \item Kuchen-Diagramm
  \begin{itemize}
    \item Gut bei Part-Whole-Relationen
    \item Tatsächliche Werte/Kategorien sind schwer zu vergleichen
  \end{itemize}
  \item Zeitreihen
  \item Sparkreihen: Zeitreihen, reduziert um Trends darzustellen
\end{itemize}


\subsection{Metaphern und Symbole}
\textbf{Metaphern} sollen Konstrukte und Konzepte verineinfacht darstellen und zugänglicher machen. Beispiele:
\begin{itemize}
  \item Städte (Cluster in Zusammenhängen)
  \item Bäume (Hierarchien)
  \item Tiere (Vererbeung)
\end{itemize}

\textbf{Symbole} stellen stark vereinfachte Darstellungen von Sachverhalten oder Objekten dar, die unter anderem auch Metaphern darstellen können (Papierkorb Icon für gelöschte Elemente).

\paragraph{Isotyp} Darstellung von statistischen Informationen durch Piktogramme (Symbole). Größe der Zahl wird durch Anzahl an Symbolen kodiert.

\subsection{Infographiken}
Infographiken sind graphische Representationen von Informationen, Daten oder Wissen, die komplexe Informationen schnell und leicht zugänglich machen sollen.

\subsubsection{Gegenüberstellung Infographik und Informationsvisualisierung}

\begin{multicols}{2}% 2-column layout
  \begin{minipage}{0.45\textwidth}
  \textbf{Infographik}
    \begin{itemize}
      \item Von Hand geschrieben
      \item Selbsterklärend
      \item Erzählt eine Geschichte
      \item Meistens Ad-Hoc
      \item Kann voreingenommen sein
    \end{itemize}
  \end{minipage}
  \textbf{Informationsvisualisierung}
    \begin{itemize}
      \item Automatisch generiert
      \item Auf verschiedene Datensätze anwendbar
      \item Nicht unbedingt selbsterklärend
    \end{itemize}
\end{multicols}

\section{Visuelle Wahrnehmung}
Visuelle Wahrnehmung ist selektiv und kann durch die Umgebung (Kontrast/Struktur) abgelenkt werden. 

\subsection{Verarbeitung visueller Informationen}
\paragraph{Dreiphasenmodell} 
\begin{enumerate}
  \item Primitive Mustererkennung
  \item Aktionen/Reflexe, Komplexe Mustererkennung
  \item Visuelles Arbeitsgedächtnis
\end{enumerate}

\paragraph{Menschliches Auge}
















\end{document}
