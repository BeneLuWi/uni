\documentclass[ngerman]{scrartcl}
\usepackage{amsmath,amsthm,amssymb}
\usepackage[T1]{fontenc}
\usepackage[utf8]{inputenc}
\usepackage{lmodern}

\usepackage{hyperref}

\title{Informations Visualisierung \\ SoSe 19}
\author{Benedikt Lüken-Winkels}
\begin{document}

\maketitle
\tableofcontents
\newpage

\section{1. Lecture}
\subsection{Orga}
\begin{itemize}
  \item Website: st.uni-trier.de/lectures/S19/IV/
  \item Tutorial: TBD (beginning: 22.-26.04.)
  \item Final exam: Do, 11.07. (elfths of July) 12-14 (H12)
\end{itemize}

\subsection{Visualisation-Basics}
\begin{itemize}
  \item Combine different kinds of information in one graphic (geographical, temporal, historical, numeric, etc.)
  \item Sharing and visualising abstract data, without physical representation 
  \item Visualisation is not:
  \begin{itemize}
    \item scientific visualisation (non-abstract data)
    \item computer graphics
    \item graphic design
  \end{itemize}
  \item \textbf{Example} Treemap
  \begin{itemize}
    \item representation of a hierarchy of a filesystem
    \item no border used for a square (compression)
    \item light effect shows curvature, indicating where the squares/areas end 
    \item $ \Rightarrow $ only 4 pixels needed instead of 9
    \item Several drawbacks (alternative: tree view)
  \end{itemize}
\end{itemize}
\paragraph{Abstract Data}
\begin{itemize}
  \item Text, table
  \item Hierarchy
  \item Composed data (Multivariate data): Example Napoleon (Slide 1)
  \item Time series: multivariate data with time as a dimension
\end{itemize}
\paragraph*{Definition: Visualisation} 
comprehend and extract data, visualisation produced automatically (not manually by humans)
\paragraph*{Visualisation process} 
\begin{itemize}
  \item graphical user interface
  \item interaction to create and manipulate the visualisation (\textbf{Visual steering})
\end{itemize}

\section{2. Lecture}

\subsection{Diagrams}
\paragraph{Pie charts}
\begin{itemize}
  \item applicable to part-whole relation
  \item Several issues 
  \begin{itemize}
    \item hard to compare values
    \item hard to compare different pie charts
  \end{itemize}
\end{itemize}
\paragraph{Other Diagrams}
\begin{itemize}
  \item Timelines
  \item Sparklines: Reduction to show trend and the change of values over time
\end{itemize}

\subsection{Metaphors and Symbols}
Make constructs/concepts more accessible/imaginable

\subsection{Symbols}
highly simplified representation of objects and acitvities

\paragraph{Isotype} Present quantity/value by number of pictograms

\subsection{Infographics}
\begin{itemize}
  \item Eyecatcher to get people interested in the presented data
  \item Contain few text
  \item Self-explanatory
  \item Should tell a \textbf{story} $ \Rightarrow $ express an opinion
\end{itemize}

\section{3. Lecture}
\subsection{Visual Memory}
\begin{itemize}
  \item The brain fills empty gaps
  \item Distraction by environment (contrast/structure)
  \item $ \Rightarrow $ visual perception is selective
\end{itemize}
\subsection{Visula Information Processing}
3 Phases of processing
\begin{enumerate}
  \item Simple patterns and colors are recognized
  \item Action system: reflexes
  \item Visual working meomry/visual query
\end{enumerate}
\subsection*{Human Eye}
Usage of the properties of visual perception (Anticipation, pattern recognition)
\begin{itemize}
  \item Eye Tracking (works by measuring the reflection form the eye's curvature)
\end{itemize}

\subsection{Color Perception}
3-Color-Theory
\begin{itemize}
  \item Each color consists of rgb
\end{itemize}
Opponent-Color-Theory
\begin{itemize}
  \item After image effect: color-receptors are getting exhausted, so white cannot be 'produced'
  \item three chemical processes with two opponent colors each 
  \item Color is perceived by the difference between the opponent colors
\end{itemize}
$ \Rightarrow $ Color and brightness are relative

\paragraph*{Design Recommendations}
\begin{itemize}
  \item Emphasize with color
  \item Differences with brightness
  \item Coding of categories: max 6 to 12 different colors
  \item Color scales should vary in color and brighntess 
  \item Color perception depends on culture
  \item Motion to grap attention/indicate a relation
  \item Strong colors/contrast can cause interta (ghost images)
\end{itemize}


\subsection{Preattentive vision}
\begin{itemize}
  \item Detect patterns before an eye movement
  \item Motion is preattentive
  \item $ \Rightarrow $ Use preattentive patterns to encode information (spot an outlier)
\end{itemize}

\subsection{Pattern Recognition}
\begin{itemize}
  \item Edge detection
  \item Simple patterns (detect small distortions)
  \item Complex patterns
  \item Object recognition (compare observation with learned patterns to recognise an object)
\end{itemize}


\subsection{Motion recognition}
Different elements perform similar motions
\begin{itemize}
  \item Recognize patterns to identify object
  \item Recognize change after each frame
  \item Movements seem related, when they are in synch
  \item $ \Rightarrow $ Indicate a relation with a synchronous animation 
  \item Motion can induce causality
\end{itemize}


\section{Lecture}
Visualization of Graphs: \textbf{Graph drawing}
\paragraph{Application}
\begin{itemize}
  \item Map-drawing: indicate multiple data sets in one map (London Underground)
  \item Ego(-centric) network: graph with personal connections 
\end{itemize}

\paragraph{Visual Encoding}
\begin{itemize}
  \item Thickness, color of edges
  \item Color of nodes
\end{itemize}

\paragraph{Asthetic Criteria}
Readability does not induce asthetic
\begin{itemize}
  \item min edge crossings
  \item min drawing
  \item min edge length
  \item min number of bends
  \item max symmetry
  \item uncover clusters
  \item max continuity amongst paths
\end{itemize}

\subsection{Layouting algorithms}
Radial Layout
\begin{itemize}
  \item fair node weight, every node's representation is equal
  \item lots of edge crossings
  \item applicable, if there is no further info about the data
\end{itemize}
Force-Directed Layout
\begin{itemize}
  \item force edges to a certain length
  \item reorder nodes
  \item try to find equilibrium, where the forces cancel out each other
\end{itemize}
Hierarchical Layout
\begin{itemize}
  \item for cyclic structures: flip the edges that close the cycle while drawing the graph
  \item depth first search provides a topological ordering of the nodes
  \item sort nodes on the lower layer until the bottom is reached, then go back to start
  \item to have a clean layout, put in dummy nodes as a spacer
\end{itemize}
Orthogonal Layout
\begin{itemize}
  \item edges follow grid (orthogonal paths)
  \item shape metrics
  \begin{itemize}
    \item describe the path the edges take by turns
    \item evaluate the paths 
  \end{itemize}
\end{itemize}
Edge Bundling
\begin{itemize}
  \item structured radial layout
  \item bundle edges with the same direction
\end{itemize}

\subsection{Matrix visualization of Graphs}
Adjacency Matrix
\begin{itemize}
  \item indicate an edge in a matrix
  \item uncovering clusters is hard
\end{itemize}
\subsubsection*{Layouting}
Compound graphs

\section{Lecture}
\subsection{Visualization of dynamic graphs}
Dynamic graph: sequence of graph states
\paragraph{Animation} see difference between layout and data changes to preserve the mental map of the graph. Examples:
\begin{itemize}
  \item TimeLine, horizontal development of the graph, vertical orientation of the graph  
  \item TimeSpiderTrees, cirular layout, each ring is one graph
  \item TimeRadarTrees, cicular layout, outer circles are a representation of the inner. The inner circle shows incoming edges, the outer shows outgoing
\end{itemize}

\subsection{Multivariate data and time series}
\paragraph{Boxplots} box showing 50 percent of data, outer borders not standardized
\paragraph{Fan Chart} wide part shows the mean (similar to the box plot) 
\paragraph{Histogram} bar represents a range of values (value ragne split into intervals)

\section{Lecture}
\subsection{Software Visualization: Architecture}
\paragraph{Pipes and Filters} Input stream providing data, putting it into a pipe of filters
\paragraph{Layered Systems} Layers provide functionality of upper layers (radial or stacked). Radial: small core, Pyramid: neutral representation 
\paragraph{Blackboard-driven} Different processes share info on one blackboard
\subsubsection{Reverse Engineering}
Create higher level of abstraction for a given system and automatically create architecture visualization. The detection of design patterns is non-trivial. To detect, the program is run and traced. 
\subsubsection{Enriched Node-Link Diagrams} 
Visuialize/Encode software metrics. Aggregation of information to simplify.
\paragraph{Class Blueprint}
Categorize methods by name and access attributes (public/protected/private...)
\paragraph{Depenecies Viewer}
Visualize package graph and dependencies between packages and methods
\paragraph{Dependency Structure Matrix DSM}
Detect cycles and indirect cycles with highlighting
\paragraph{Software Citites and Maps}
2D plane represents system. Hierarchy shown with trees/dimesions. 3rd dimension can be used to show other metrics, like evolution/age/dependencies
\paragraph{Summary}
Ad-hoc diagrams hard to understand without explanation. With reverse engineering automatic creation for specific techniques are possible	 

\section{Lecture}
\subsection{Dynamic Program Visualization}

\paragraph{Dynamic Data Acquisition} invasive mehtod, monitoring the behaviour of a program before/after each instruction. Might alter the program execution. 
NACHLESEN

\section{Lecture}
\subsection{Visula Debugging}
Slices are parts/slices of the huge dependency graph in a program 
\paragraph{Static Slice} How is a variable changed by other code points. Slice is a samll part
\paragraph{Dynamic Slice} 
\paragraph{Execution Slices} Sequence of program points.

\paragraph{Dice} Difference of two Slices.

\paragraph{X-Slice} (Heuristic) Compare a run with failing and compiling input. Only the failing program points are highlighted. Color coding coverage data by failure propability and evidence for failure.

\paragraph{Test Blueprint} Highlight non-executed program points in the Class Blueprint. 


\subsection{Software Evolution}
aka Software Development Process $ \Rightarrow $ Software changes in its lifetime.

\paragraph{Software Archive} version control/collection of the history of a program of any kind. 

\paragraph{Color-coding}
\begin{itemize}
  \item Line Representation: indentation/different metrices
  \item Code Age: when was a file/line changed
  \item Pixel Representation
  \item Version-specific Code: highlight eg platform specific code
  \item Depth of nested blocks
  \item CVSScan: different versions for a file with LOC as bar height.
\end{itemize}

\paragraph{Evolution Matrix} Classes are represented as boxes. Box height and width encode a certain metric. $ \Rightarrow $ No insight on program structure

\paragraph{Call Graph} Which function calls wich function (low level info). Encode program structure. Edge splatting (the more often an edge is drawn the more intense it color gets) shows call clusters.

\subsubsection{Visual Data Mining in Software Architecture}
\paragraph{Data Mining Process}
Startgin with a version control program (git)
\begin{enumerate}
  \item Analysis
  \item Extraction
  \item Data Mining
  \item Visual Data Mining
\end{enumerate} 


\paragraph{Coupling}
\begin{itemize}
  \item Evolutionary Coupling artifact are related, when they are changed together.
  \item Logical Coupling artifacts are related, when they are programmatically calling each other.
\end{itemize}







\end{document}











