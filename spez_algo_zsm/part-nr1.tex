\section{Union-Find}
	Beschreiben Sie jeweils eine Lösung für das Union-Find-Problem mit Laufzeit
	\begin{enumerate}
		\item $O(\log n)$ (amortisiert) für UNION und $O(1)$ für FIND
		\item $O(1)$ für UNION und $O(\log n)$ für FIND
	\end{enumerate}	
	wobei $n$ die Anzahl der Elemente ist. Begründen Sie in beiden Fällen die entsprechenden Laufzeiten.

	\subsection*{Lösung 1.)} Union in $ O(\log n) $, Find in $ O(1) $. \textbf{Idee:} Relable the smaller half, sodass jedes Element nur maximal $ \log n $ geändert wird:
	\paragraph{Datenstruktur}
	\begin{verbatim}
	    name[x]: Name des Blocks, der x enthält
	    size[A]: Größe des Blocks A (Init 1)
	    list[A]: Liste der Elemente in Block A
	\end{verbatim}
	    \vspace{5mm}
    \begin{algorithm}[H]
    \SetAlgoLined
    \ForEach{x $\in N $}{
        name[x]$\gets$ x\;
        size[x]$\gets$ 1\;
        list[x]$\gets$ \{x\}\;
    }
    \caption{Initialisierung}
    \end{algorithm}
        \vspace{5mm}
    \begin{algorithm}[H]
    \SetAlgoLined
    \Return name[x]\;
    \caption{Find(x)}
    \end{algorithm}
        \vspace{5mm}
    \begin{algorithm}[H]
    \eIf{size[A] $\geq$ size[B] }{
        \ForEach{ x $\in$ list[B]}{
            name[x] $\gets$ A\;
        }
        size[A] $ \gets $ size[A]+size[B]\;
        list[A].append(list[B])\;     
    }{
        \ForEach{ x $\in$ list[A]}{
           name[x] $\gets$ B\;
        }
        size[B] $ \gets $ size[A]+size[B]\;
        list[B].append(list[A])\;       
    
    }

    \caption{Union(A,B)}
    \end{algorithm}
    
    \paragraph{Laufzeit}
    \begin{itemize}
        \item[] Find: $ O(1) $. Lookup im Array.
        \item[] Union: $ O(\log n) $. Jedes x kann maximal $ \log n $ mal seinen Namen ändern, da es sich nach jeder Namensänderung in einer doppelt so großen Liste befindet.
    \end{itemize}
    
    
    \subsection*{Lösung 2.)}
    Union in $ O(1) $ und Find in $ O(\log n) $. \textbf{Idee:} Bei Union Anhängen des kleineren Teilbaums an den Größeren.\\
    Das ergibt die Abschätzung size[x] $ \geq 2^{höhe(x)} $, bzw $\log_2$(size[x]) $\geq$ höhe(x), also wird der Baum nie tiefer, als $ \log n $
    \paragraph{Datenstruktur}
    \begin{verbatim}
        name[x]: Name des Blocks mit Wurzel x (nur, wenn x eine Wurzel relevant)
        size[x]: Anzahl der Knoten im Unterbaum mit Wurzel x
        wurzel[x]: Wurzel des Blocks mit Namen x
        vater[x]: Vaterknoten des Knotens x. 0, wenn x Wurzel
    \end{verbatim}
    \begin{algorithm}[H]
    \SetAlgoLined
    \ForEach{x $\in N $}{
        name[x]$\gets$ x\;
        size[x]$\gets$ 1\;
        wurzel[x]$\gets$ x\;
        vater[x]$\gets$ 0\;
    }
    \caption{Initialisierung}
    \end{algorithm}
    \vspace{5mm}
    \begin{algorithm}[H]
    \SetAlgoLined
    \While{vater[x] $ \neq $ 0}{
        x $ \gets$ vater[x] \;
    }
    \Return name[x]\;
    \caption{Find(x)}
    \end{algorithm}
    \vspace{5mm}
    \begin{algorithm}[H]
    \SetAlgoLined
    a $ \gets $ wurzel[A]\;
    b $ \gets $ wurzel[B]\;
    \eIf{size[a] $\geq $ size[b]}{
        vater[b] $\gets$ a\;
        name[a] $\gets$ C\;
        wurzel[C] $\gets$ a\;
        size[a] $\gets$ size[a] + size[b]\;
    } {
        analog\;
    }
    \caption{Union(A, B, C)}
    \end{algorithm}
    
    \paragraph{Laufzeit}
    \begin{itemize}
        \item[] Find: $ O(\log n) $. Höhe des Baums bleibt maximal $ \log n $, da der kleinere Teilbaum immer an die Wurzel des Größeren gehangen wird und sich die Tiefe des Baums durch seine Größe abschätzen lässt.
        \item[] Union: $ O(1) $. Lediglich die Wurzel muss umgeschrieben werden.
    \end{itemize}
    
    
    
    
    
    
    
    
    
